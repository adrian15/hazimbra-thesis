% Initial Pacemaker Setup


\chapter{Initial Pacemaker Setup}
This chapter is about the Pacemaker initial setup that it's needed so that Zimbra can be treated as a resource.

\section {About OCF Resource agents}
A resource agent is a standardized interface for a cluster resource. In translates a standard set of operations into steps specific to the resource or application, and interprets their results as success or failure. An OCF resource agent is based on the Open Clustering Framework Resource Agent API specifications. 

TODO:
\begin{verbatim}
http://linux-ha.org/wiki/Resource_Agents
http://linux-ha.org/wiki/OCF_Resource_Agents
\end{verbatim}



\section {bTactic Zimbra OCF installation}
We need to download the Zimbra OCF script from bTactic ORG site.

TODO:
\begin{verbatim}
http://www.btactic.org
http://www.btactic.org/btactic_ocf_0.0.2.tar.gz
\end{verbatim}
.

From the tar.gz we will use the zimbra script which we will copy into:
\begin{verbatim}
/usr/lib/ocf/resource.d/btactic
\end{verbatim}
.

So in order to install it we enter into a temporary directory:
\begin{verbatim}
mkdir /tmp/temp
cd /tmp/temp
\end{verbatim}
Download and extract it:
\begin{verbatim}
wget "http://www.btactic.org/btactic_ocf_0.0.2.tar.gz"
tar xzf btactic_ocf_0.0.2.tar.gz
\end{verbatim}
Make the btactic resource directory and copy zimbra file in there:
\begin{verbatim}
mkdir --parents /usr/lib/ocf/resource.d/btactic
cp zimbra /usr/lib/ocf/resource.d/btactic
\end{verbatim}
We finally make sure to give the script executable permissions:
\begin{verbatim}
chmod +x /usr/lib/ocf/resource.d/btactic/*
\end{verbatim}

