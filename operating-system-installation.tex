% Operating System installation


\chapter{Operating System installation}
\label{chap:operating-system-installation}
This chapter explains the Operating System installation.

\section {Introduction}
We will use Ubuntu 12.04 in its 64bit mode because it is one of the official supported Operating System for Zimbra 8 versions. We denote an external DRBD metadata as DRBD-Meta-Disk. We can understand it is an special partition that helps DRBD system to track changes between synchronised partitions between both primary and secondary nodes. We can find a more accurate definition at Linbit site: \cite{LinbitDRBDInternals}.

These instructions are valid for both primary and secondary nodes. The only difference is that each one of them will have a different hostname.

An Ubuntu Gnu/Linux installation might be as complex as of having Logical Volumes Group (LVM) for the ease of hard disk space management. In this installation we will avoid that kind of partitions so that final Pacemaker setup (Chapter \ref{chap:pacemaker-final-setup}) is easier.

\section {Ubuntu 12.04 64 bit minimal}
In order to track all the requisites and just install what the high availability system needs we will use an Ubuntu minimal disk for installation. These disks can be downloaded from \cite{UbuntuMinimalDisk}.

The used download was: \textit{Ubuntu 12.04 "Precise Pangolin" Minimal CD} from the \textit{64-bit PC (amd64, x86\_64)} section. 

\subsection {Installer boot menu}
We just press Return key to select default boot option: \textit{Install}.
\subsection {Select a language}
We choose the desired language: \textit{English}
\subsection {Select your location}
We choose our location: \textit{United States}
\subsection {Configure the keyboard}
We choose \textit{no} for semi automated keyboard detection. That let us choose \textit{English (US} as country of origin for the keyboard. Finally we select default \textit{English (US)} keyboard.
\subsection {Network}
The network detection will be automatically sorted out if one our of interfaces is bridge in a network provided of a DHCP server. Once we boot into the operating system we will setup network manually. We are asked our hostname. We choose: \textit{primary} in case we are installing primary node and \textit{secondary} in case we are installing secondary node.
\subsection {Ubuntu archive mirror country}
We select default \textit{United States} and its associated mirror: \textit{us.archive.ubuntu.com}. When asked for HTTP proxy information we just press Return key to continue.
\subsection {Checking Ubuntu mirror}
Ubuntu installation will perform some background checkings and downloads without the user being noticed. Then the installation suddenly continues by retrieving and installing additional packages and components.
\subsection {Set up users and passwords}
We first are asked to write \textit{Full name for the new user} and then \textit{Username for the account}. We will provide both of them. We are also prompted twice the user password. When asked we choose not to encrypt the home directory.
\subsection {Configure the clock}
Given our detected physical location we just press Return key to validate it.
\subsection {Partition disks}
\subsubsection {Introduction}
Assuming a 1.8 Terabyte hard disk in order to setup DRBD-Meta-Disk there's enough with 59 megabytes. We will be on the safe side and setup it with a 150 megabytes size. In order to safe calculate other DRBD meta disk partitions we can check \cite{LinbitDRBDInternals}.

We will not be using a SWAP partition because Zimbra works better without it. If a SWAP partition would be needed to be used we encourage to add it as a fixed size secondary hard disk.

TODO: According to Zimbra Performance Tunning Guidelines

\subsubsection {Manual partitioning}
We select \textit{Manual} partitioning method. We then select our hard disk in our case: \textit{SCSI1 (0,0,0) (sda) - 107.4 GB ATA VBOX HARD DISK}. When asked to create a new empty partition table on the device we reply \textit{Yes}.
Now we are back at Partition disk screen. For each one of the partitions to be created we should select \textit{FREE SPACE} under our hard disk. Then select create a new partition.  Once a partition has been defined we will confirm its settings by selecting \textit{Done setting up the partition}. We will describe in an schematic manner how the partitions should be created.

\textbf{Root partition}
\begin{itemize}
  \item New partition size: \textit{10 GB}
  \item Type for the new partition: \textit {Primary}
  \item Location for the new partition: \textit{Beginning}
\end{itemize}
Partition settings for Root partition will be left as default except for the Bootable flag which should be set to \textit{on}. These were:
\begin{itemize}
  \item Use as: \textit{Ext4 Journaling file system}
  \item Mount point: \textit{/}
  \item Mount options: \textit{defaults}
  \item Label:	\textit{none}
  \item Reserved blocks: \textit{5\%}
  \item Typical usage: \textit{standard}
  \item Bootable flag: \textbf{\textit{on}}
\end{itemize}


\textbf{DRBD-Meta-Disk partition}
\begin{itemize}
  \item New partition size: \textit{150 MB}
  \item Type for the new partition: \textit {Primary}
  \item Location for the new partition: \textit{Beginning}
\end{itemize}
Partition settings for DRBD-Meta-Disk partition were finally:
\begin{itemize}
  \item Use as: \textit{do not use}
  \item Bootable flag: \textit{off}
\end{itemize}

\textbf{ZimbraData partition}
\begin{itemize}
  \item New partition size: \textit{Rest of disk space}
  \item Type for the new partition: \textit {Primary}
  \item Location for the new partition: \textit{Beginning}
\end{itemize}
Partition settings for ZimbraData partition were finally:
\begin{itemize}
  \item Use as: \textit{Ext4 Journaling file system}
  \item Mount point: \textbf{\textit{/opt}}
  \item Mount options: \textbf{\textit{defaults}}
  \item Label:	\textit{none}
  \item Reserved blocks: \textit{0\%}
  \item Typical usage: \textit{standard}
  \item Bootable flag: \textit{off}
\end{itemize}
We are using \textit{/opt} partition for Zimbra Data because Zimbra programs and its data is stored in \textbf{/opt/zimbra} by default.

Now we are going to confirm all our created partitions by selecting \textit{Finish partitioning and write changes to disk}.
When asked to return to the partitioning menu because a swap partition is needed we will just skip it by saying: \textit{No}. Finally we confirm that we want to write changes to disk by selecting: \textit{Yes}.

\subsection {Configuring x11-common}
We select: \textit{No automatic updates} when asked how to manage upgrades just to make things easier. In a production environment we must select \textit{Install security updates automatically}.

\subsection {Software selection}
When asked which software to install we will only check \textit{OpenSSH server} for installation in order to keep the installation as minimal and functional as possible.

\subsection {Install the GRUB}
When asked to Install teh GRUB boot loader to master boot record we select Yes.

\subsection {Finish the installation}
When we are asked if the system clock is set to UTC we select Yes. Once clock it's been set we need to do a change in Virtualbox. In the running Virtual machine we select \textit{Devices} menu, CD/DVD devices submenu and then we uncheck the minimal Ubuntu 12.04 64 bit iso.

If asked we force unmount in Virtualbox.

Finally we click on \textit{Continue} to finish the installation.
