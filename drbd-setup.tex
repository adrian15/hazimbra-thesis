% DRBD Setup


\chapter{DRBD Setup}
\label{chap:drbd-setup}
This chapter explains the DRBD Setup.

\section {Introduction}
DRBD is a system that let us mirror a block device via an assigned device. DRBD can be understood as network based raid-1 and it's used in high availability (HA) clusters (\cite{LinbitDRBDWhatIs}).
We're using DRBD to mirror the block device where Zimbra files will be stored on.

\section {Requirements}
We will install DRBD packages for the Ubuntu 12.04 system thanks to the following command:
\begin{verbatim}
sudo apt-get install drbd8-utils
\end{verbatim}
in both nodes.
\section {Communication hosts}
For the purpose of both DRBD and Corosync communication we need to define our hosts from the communication link point of view. We need to edit \textit{/etc/hosts} and make sure we have \textbf{in both nodes}:

\begin{verbatim}
10.0.66.201 primary.zimbraha.lan primary
10.0.66.202 secondary.zimbraha.lan secondary
\end{verbatim}

\section {Zimbra stop}
As Zimbra is using the DRBD backing device which it is \textit{/dev/sda3} we need to unmount it and that means that we need to stop Zimbra services before doing so.

So we will first stop zimbra on \textbf{primary node} by doing:
\begin{verbatim}
service zimbra stop
\end{verbatim}

And then in \textbf{both nodes} we will umount the partition with:
\begin{verbatim}
sudo umount /dev/sda3
\end{verbatim}

\section {DRBD Resource config}
\textbf{To be performed in both nodes}.

We will backup main drbd configuration file:
\begin{verbatim}
cp /etc/drbd.conf /etc/drbd.conf.orig
\end{verbatim}
.

We then need to edit:
\begin{verbatim}
/etc/drbd.conf
\end{verbatim}
so that it has:
\begin{verbatim}
include "drbd.d/global_common.conf";
include "drbd.d/*.res";

resource zimbradata {
  protocol C;
  handlers {
    pri-on-incon-degr "halt -f";
  }
  startup {
    degr-wfc-timeout 120; # 2 min
  }
  disk {
    on-io-error detach;
  }
  net {
  }
  syncer {
    rate 10M;
    al-extents 257;
  }
  on primary {
    device /dev/drbd0;
    disk /dev/sda3;
    address 10.0.66.201:7788;
    meta-disk /dev/sda2[0];
  }
  on secondary {
    device /dev/drbd0;
    disk /dev/sda3;
    address 10.0.66.202:7788;
    meta-disk /dev/sda2[0];
  }
}
\end{verbatim}
.

TODO: Explain drbd.conf contents and what they mean more or less.

If we want to modify our drbd.conf to increase sync rate, synchronised devices or any other settings we can check documentation at \cite{LinbitDRBDdrbdconf}.

\section {Start DRBD module}
\textbf{To be performed in both servers}.
\begin{verbatim}
modprobe drbd
\end{verbatim}
.
\section {Metadata disk initialisation}
\textbf{To be performed in both servers}.
We make sure the metadata partition does not have any prior metadata signature.
\begin{verbatim}
dd if=/dev/zero of=/dev/sda2 bs=1K count=100
\end{verbatim}
And we create the zimbradata metadata partition:
\begin{verbatim}
drbdadm create-md zimbradata
\end{verbatim}
.

The output for both servers will similar to:
\begin{verbatim}
Writing meta data...
initializing activity log
NOT initialized bitmap
New drbd meta data block successfully created.
success
\end{verbatim}
if sucedeed.

\section {First DRBD synchronisation}
\textbf{To be performed in both servers}.
\begin{verbatim}
drbdadm up all
\end{verbatim}
If everything goes ok we should return to the prompt.

Were we will asked about usage, we just reply that we don't want to participate in the survey by saying 'no'.

\textbf{To be performed in Primary server only}.
\begin{verbatim}
drbdadm -- --overwrite-data-of-peer primary all
drbdadm -- connect all
\end{verbatim}
.

If we happen to find a \textit{net-config disconnect first} error we can safely ignore it.

In order to check DRBD first synchronisation status we need to check /proc/drbd file with:
\begin{verbatim}
cat /proc/drbd
\end{verbatim}
which will output something like:
\begin{verbatim}
version: 0.7.20 (api:77/proto:74)
SVN Revision: 1743 build by <a href="mailto:phil@mescal">\
phil@mescal</a>, 2005-01-31 12:22:07
0: cs:SyncSource st:Primary/Secondary ld:Consistent
ns:13441632 nr:0 dw:0 dr:13467108 al:0 \
bm:2369 lo:0 pe:23 ua:226 ap:0
[==>..............] sync'ed: 3.1% (7000/7168)M
finish: 1:14:16 speed: 2,644 (2,204) K/sec
1: cs:Unconfigured
\end{verbatim}
.

We must wait for the first synchronisation to end so that we can safely complete the rest of the instructions.

Final end output will similar to:
\begin{verbatim}
version: 8.3.11 (api:88/proto:86-96)
srcversion: 93CE421BB73A731BDC72D8E 
 0: cs:Connected ro:Primary/Secondary ds:UpToDate/UpToDate C r-----
    ns:5691029 nr:0 dw:0 dr:6002290 al:0 bm:367 lo:0 pe:0 ua:0 ap:0 ep:1 wo:f oos:0
\end{verbatim}
where we can see that both primary and secondary are updated (\textit{UpToDate}).

