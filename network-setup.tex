% Network Setup


\chapter{Network setup}
\label{chap:network-setup}
This chapter explains the network setup.

\section {Network schema}
We can just check the High Availability main schema (figure \ref{fig:main-schema}) where network has been already described. There are three networks. The service link is the main network interface for serving content to the final users. The communication link, which is used for the cluster management and syncronisation is done via a crossover cable. Finally NAT link gives the machines access to Internet.

One of the reason why there are two links is because we use communication link for drbd synchronisation and that avoids DRBD traffic to both service traffic. The other reason is because cluster communication is not interfered by service traffic and that avoids false nodes being offline detections.

For more detail on how it's implemented in Virtualbox we can check \textbf{\ref{sec:virtualbox-implementation} Virtualbox implementation} chapter.

\section {Network setup}

\subsection {High Availability service}
\subsubsection {FQDN}
High Availability server FQDN will be:
\begin{verbatim}
public.zimbraha.lan
\end{verbatim}

\subsubsection {Service link}
Service link network setup consists of a Class C configuration where the network card address is 192.168.77.203, as per being a Class C its netmask is 255.255.255.0 and thus its broadcast is 192.168.77.255.

This interface will not be configured in \textit{/etc/network/interfaces} but by pacemaker itself thanks to its cluster definition.

That ip will be the one that email client will connect to our service. Our cluster will make sure that it's only configured in only one server, the active one.

\subsection {Primary server}

\subsubsection {Communication link}
Primary server's Communication link network setup consists of a Class C configuration where the network card address is 10.0.66.201, as per being a Class C its netmask is 255.255.255.0 and thus its broadcast is 10.0.66.255. As a gateway it will be using the first network address in the network range which is 10.0.66.1.


The correspondent configuration code for \textit{/etc/network/interfaces} file is:
\begin{verbatim}
auto eth0
iface eth0 inet static
        address 10.0.66.201
        netmask 255.255.255.0
        broadcast 10.0.66.255
\end{verbatim}
\subsection {FQDN}
Fully Qualified Domain Name for primary server will be:
\begin{verbatim}
primary.zimbraha.lan
\end{verbatim}
\subsection {Additional links}
In order to have Internet connectivity an additional link set as a NAT interface in Virtualbox will be added.

\subsection {Secondary server}

\subsubsection {Communication link}
Secondary server's Communication link network setup consists of a Class C configuration where the network card address is 10.0.66.202, as per being a Class C its netmask is 255.255.255.0 and thus its broadcast is 10.0.66.255. As a gateway it will be using the first network address in the network range which is 10.0.66.1.

The correspondent configuration code for \textit{/etc/network/interfaces} file is:
\begin{verbatim}
auto eth1
iface eth1 inet static
        address 10.0.66.202
        netmask 255.255.255.0
        broadcast 10.0.66.255
\end{verbatim}
\subsection {FQDN}
Fully Qualified Domain Name for secondary server will be:
\begin{verbatim}
secondary.zimbraha.lan
\end{verbatim}
\subsection {Additional links}
In order to have Internet connectivity an additional link set as a NAT interface in Virtualbox will be added.

\section {Firewall}
Our default Ubuntu minimal installation doesn't have 
\subsection {Zimbra ports}
These are the Zimbra ports that would need to be open in a production evironment (\cite{ZimbraWikiPorts}):
\begin{itemize}
  \item 25 smtp [mta] - incoming mail to postfix
  \item 80 http [mailbox] - web mail client
  \item 110 pop3 [mailbox]
  \item 143 imap [mailbox]
  \item 443 https [mailbox] - web mail client over ssl
  \item 465 smtps [mta] - incoming mail to postfix over ssl (Outlook only)
  \item 587 smtp [mta] - Mail submission over tls
  \item 993 imaps [mailbox] - imap over ssl
  \item 995 pops [mailbox] - pop over ssl
  \item 7071 https [mailbox] - admin console
\end{itemize}
And these are the Zimbra ports typically only used by the zimbra system itself (\cite{ZimbraWikiPorts}).
\begin{itemize}
  \item 389 ldap [ldap]
  \item 636 ldaps [ldaps, if enabled]
  \item 7025 lmtp [mailbox] - local mail delivery
  \item 7047 conversion server
  \item 7306 mysql [mailbox]
  \item 7307 mysql [logger] - logger
  \item 7780 http [mailbox] - spell check
  \item 10024 smtp [mta] - to amavis from postfix
  \item 10025 smtp [mta] - back to postfix from amavis
\end{itemize}
System access ports (\cite{ZimbraWikiPorts}) are:
\begin{itemize}
  \item 22 ssh
  \item 514 syslogd [logger] (udp)
\end{itemize}

\subsection {High Availability ports}
In the service link interfaces we need to make sure these ports are accesible from one node to another:
\begin{itemize}
  \item Corosync communication port: 5405 (udp)
  \item Corosync communication port: 5404 (udp)
  \item DRBD: 7788
\end{itemize}


