% Zimbra installation


\chapter{Zimbra installation}
\label{chap:zimbra-installation}
This chapter explains the Zimbra OSE installation.

\section {Introduction}
Zimbra is an enterprise-class email, calendar and collaboration solution, built for the cloud, both public and private. With a redesigned browser-based interface, Zimbra offers the most innovative messaging experience available today, connecting end users to the information and activity in their personal clouds (\cite{ZimbraWeb}).

One of the main Zimbra services is a web server where the final user can check its own email, calendar and contacts among others. This is the service that we will check when testing high availability.

\section {Operating system checks}
\subsection {/etc/hosts}
\textbf{On both nodes}
We will make sure in \textit{/etc/hosts} file that we have:

\begin{verbatim}
192.168.77.203 public.zimbraha.lan public
\end{verbatim}
\subsection {/etc/hostname}
\textbf{On primary node} we will make sure that \textit{/etc/hostname} contains:

\begin{verbatim}
primary
\end{verbatim}
. In order to apply changes we will run:
\begin{verbatim}
service hostname restart
\end{verbatim}

Same operation but using \textit{secondary} will need to be performed \textbf{on secondary node}.

\section {Package requirements}
\textbf{On both nodes} we will make sure we meet the Zimbra system package requirements.

\begin{verbatim}
sudo apt-get install libgmp3c2 libexpat1 \
libstdc++6 sysstat libpcre3 libperl5.14 \
sqlite3 libidn11 pax
\end{verbatim}

\section {Zimbra 8.0.4 for Ubuntu 12.04}
\textbf{On both nodes}
Zimbra 8.0.4 for Ubuntu 12.04 in form of a tar.gz file was downloaded from \cite{Zimbra8Download}.
Once downloaded is advised to check its md5sum. Finally we untar it and cd into the untarred directory.

First we download tgz file and its md5sum file:
\begin{verbatim}
wget "http://files2.zimbra.com\
/downloads/8.0.4_GA/zcs-8.0.4\
_GA_5737.UBUNTU12_64\
.20130524120036.tgz.md5"

wget "http://files2.zimbra.com\
/downloads/8.0.4_GA/zcs-8.0.4\
_GA_5737.UBUNTU12_64\
.20130524120036.tgz"
\end{verbatim}
Then we check if the download is correct by calculating its md5sum:
\begin{verbatim}
md5sum -c zcs-8.0.4\
_GA_5737.UBUNTU12_64\
.20130524120036.tgz.md5
\end{verbatim}
If md5sum is correct you will see:
\begin{verbatim}
zcs-8.0.4_GA_5737.UBUNTU12_64.20130524120036.tgz: OK
\end{verbatim}
Finally we untar it:
\begin{verbatim}
tar xzf zcs-8.0.4\
_GA_5737.UBUNTU12_64\
.20130524120036.tgz
\end{verbatim}

\section {\label{sec:complete-install-script-on-primary}Complete Install script on Primary}
\subsection {Service link manual configuration}
We need to configure service link manually with:
\begin{verbatim}
ifconfig eth0 192.168.77.203 \
netmask 255.255.255.0
\end{verbatim}

\subsection {Installation start}
\textbf{Only on primary node} we will change directory into Zimbra installation directory and run installation script:

\begin{verbatim}
cd zcs-8.0.4_GA_5737.UBUNTU12_64.20130524120036
./install.sh
\end{verbatim}
\subsection {License agreement}
We will have to agree with the terms of the software license agreement by typing \textit{Yes} and pressing Return key.

\subsection {\label{subsec:zimbra-packages-install}Zimbra packages install}
Now we are requested which Zimbra packages we want to install. We will choose the default ones.

\begin{verbatim}
Select the packages to install

Install zimbra-ldap [Y] 

Install zimbra-logger [Y] 

Install zimbra-mta [Y] 

Install zimbra-snmp [Y] 

Install zimbra-store [Y] 

Install zimbra-apache [Y] 

Install zimbra-spell [Y] 

Install zimbra-memcached [N] 

Install zimbra-proxy [N] 
\end{verbatim}

We are told that the system will be modified and if we want to continue. We reply \textit{Yes}.

Zimbra packages installation is being performed.

\subsection {Change hostname}
When asked:
\begin{verbatim}
DNS ERROR resolving primary.zimbraha.lan
It is suggested that the hostname be resolvable via DNS
Change hostname [Yes]
\end{verbatim}
we will change it (\textit{Yes}) so that it says: \textit{public.zimbraha.lan}.

As DNS isn't setup the same question will be asked. We will answer that \textit{No} we don't want to re-enter hostname.

In a production environment we will need to make sure an A DNS field so that public.zimbraha.lan resolves to its public ip or internal ip.

\subsection {Change domain name}
When asked:
\begin{verbatim}
DNS ERROR resolving MX for public.zimbraha.lan
It is suggested that the domain name have an MX record configured in DNS
Change domain name? [Yes] 
\end{verbatim}
we will change the domain name and specify \textit{zimbraha.lan}. As DNS isn't setup the same question will be asked. We will answer that \textit{No} we don't want to re-enter domain name.

\subsection {Set password and apply}
We are shown the zmsetup script:
\begin{verbatim}
Main menu

   1) Common Configuration:                                                  
   2) zimbra-ldap:                             Enabled                       
   3) zimbra-store:                            Enabled                       
        +Create Admin User:                    yes                           
        +Admin user to create:                 admin@zimbraha.lan            
******* +Admin Password                        UNSET                         
        +Anti-virus quarantine user:           virus-quarantine.
						  uthvrv6amf@zimbraha.lan
        +Enable automated spam training:       yes                           
        +Spam training user:                   spam.aw9h_i7ms@zimbraha.lan   
        +Non-spam(Ham) training user:          ham.djdehqsd@zimbraha.lan     
        +SMTP host:                            public.zimbraha.lan           
        +Web server HTTP port:                 80                            
        +Web server HTTPS port:                443                           
        +Web server mode:                      https                         
        +IMAP server port:                     143                           
        +IMAP server SSL port:                 993                           
        +POP server port:                      110                           
        +POP server SSL port:                  995                           
        +Use spell check server:               yes                           
        +Spell server URL:                     http://public.
						  zimbraha.lan:7780/aspell.php
        +Configure for use with mail proxy:    FALSE                         
        +Configure for use with web proxy:     FALSE                         
        +Enable version update checks:         TRUE                          
        +Enable version update notifications:  TRUE                          
        +Version update notification email:    admin@zimbraha.lan            
        +Version update source email:          admin@zimbraha.lan            

   4) zimbra-mta:                              Enabled                       
   5) zimbra-snmp:                             Enabled                       
   6) zimbra-logger:                           Enabled                       
   7) zimbra-spell:                            Enabled                       
   8) Default Class of Service Configuration:                                
   r) Start servers after configuration        yes                           
   s) Save config to file                                                    
   x) Expand menu                                                            
   q) Quit
\end{verbatim}
we will type \textit{3} in order to select \textit {zimbra-store} configuration.

We are shown Store configuration:
\begin{verbatim}
Store configuration

   1) Status:                                  Enabled                       
   2) Create Admin User:                       yes                           
   3) Admin user to create:                    admin@zimbraha.lan            
** 4) Admin Password                           UNSET                         
   5) Anti-virus quarantine user:              virus-quarantine.
						  uthvrv6amf@zimbraha.lan
   6) Enable automated spam training:          yes                           
   7) Spam training user:                      spam.aw9h_i7ms@zimbraha.lan   
   8) Non-spam(Ham) training user:             ham.djdehqsd@zimbraha.lan     
   9) SMTP host:                               public.zimbraha.lan           
  10) Web server HTTP port:                    80                            
  11) Web server HTTPS port:                   443                           
  12) Web server mode:                         https                         
  13) IMAP server port:                        143                           
  14) IMAP server SSL port:                    993                           
  15) POP server port:                         110                           
  16) POP server SSL port:                     995                           
  17) Use spell check server:                  yes                           
  18) Spell server URL:                        http://public.
						  zimbraha.lan:7780/aspell.php
  19) Configure for use with mail proxy:       FALSE                         
  20) Configure for use with web proxy:        FALSE                         
  21) Enable version update checks:            TRUE                          
  22) Enable version update notifications:     TRUE                          
  23) Version update notification email:       admin@zimbraha.lan            
  24) Version update source email:             admin@zimbraha.lan            

Select, or 'r' for previous menu [r] 
\end{verbatim}
we will select \textit{4} for \textit{Admin password}.

We're shown:
\begin{verbatim}
Password for admin@zimbraha.lan (min 6 characters): [31I2YlBw3]
\end{verbatim}

We will just press enter to accept suggested password.

Now setup is complete:
\begin{verbatim}
Main menu

   1) Common Configuration:                                                  
   2) zimbra-ldap:                             Enabled                       
   3) zimbra-store:                            Enabled                       
   4) zimbra-mta:                              Enabled                       
   5) zimbra-snmp:                             Enabled                       
   6) zimbra-logger:                           Enabled                       
   7) zimbra-spell:                            Enabled                       
   8) Default Class of Service Configuration:                                
   r) Start servers after configuration        yes                           
   s) Save config to file                                                    
   x) Expand menu                                                            
   q) Quit                                    

*** CONFIGURATION COMPLETE - press 'a' to apply
Select from menu, or press 'a' to apply config (? - help) 

\end{verbatim}


We press \textit{r} to return to previous menu and finally we apply the configuration by pressing \textit{a}.

We're asked to save configuration data to a file and we are offered a file name for it. We will just accept defaults:
\begin{verbatim}
Save configuration data to a file? [Yes] 
Save config in file: [/opt/zimbra/config.21624] 
Saving config in /opt/zimbra/config.21624...done.
\end{verbatim}

When informed that the system will be modified we will just said that \textit{Yes} we want to continue.

\begin{verbatim}
The system will be modified - continue? [No] 
\end{verbatim}

\subsection {Zimbra notification}
We are asked to notify Zimbra of our installation. We will reply \textit{No}.

\begin{verbatim}
You have the option of notifying Zimbra of your installation.
This helps us to track the uptake of the Zimbra Collaboration Server.
The only information that will be transmitted is:
        The VERSION of zcs installed (8.0.4_GA_5737_UBUNTU12_64)
        The ADMIN EMAIL ADDRESS created (admin@zimbraha.lan)

Notify Zimbra of your installation? [Yes]
\end{verbatim}

\subsection {End of installation}
Installation ends with this message where we can just press Return.

\begin{verbatim}
Configuration complete - press return to exit
\end{verbatim}

\subsection {\label{service-link-disable}Service link disable}
In order to perform dummy installation on Secondary we need service link to be configured manually on secondary node. That conflicts with service link being configured on primary node. We will also need to stop Zimbra services.

We are going to disable service link on primary with:
\begin{verbatim}
service zimbra stop
ifconfig eth0 down
\end{verbatim}

\section {Dummy installation on Secondary}
We will perform the same exact installation steps as the ones described in 
\textbf{\ref{sec:complete-install-script-on-primary} Complete Install script on Primary} even the \textbf{\ref{service-link-disable} Service link disable} subsection which we need because the Cluster will be the only one to configure the service link.

Now we are going to delete our unused Zimbra installation \textbf{only on secondary node} with:
\begin{verbatim}
rm -rf /opt/zimbra
\end{verbatim}

As this is a dummy installation we don't need to write down Zimbra Administration password.