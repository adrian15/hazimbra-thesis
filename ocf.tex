% OCF


\chapter{Zimbra OCF Resource Agent development}
\label{chap:zimbra-ocf}
This chapter explains the development of a Zimbra OCF Resource Agent.

\section {Introduction}
A resource agent is a standardized interface for a cluster resource. It translates a standard set of operations into steps specific to the resource or application, and interprets their results as success or failure (\cite{ResourceAgentsWiki}). An OCF resource agent is based on the Open Clustering Framework Resource Agent API specifications(\cite{OCFResourceAgentsWiki}).

In order to use the latest Pacemaker version to manage Zimbra HA an OCF Resource Agent script was needed. So a Zimbra OCF Resource Agent was developed.

\section {Development log}
\begin{itemize}
  \item {In order to develop a resource agent it is advised to have cluster management stopped with:
\begin{verbatim}
service corosync stop
\end{verbatim}
and to handle manually the resources which current script is dependant on.
  }
  \item {It is advised to use \textit{ocf-tester} script which will be found in \textit{cluster-agents} package in order to debug if the OCF script actions comply with the required ones in \cite{OCFResourceAgentsWiki}.
  }
  \item {The minimal required actions were implemented because Zimbra server does not promote or demote itself.
  }
  \item {In order to validate the tests with ocf-tester you can avoid generating valid meta-data XML output. However when using it in production XML output has to be a valid meta-data XML. 
  }
  \item {Default times for waiting to Zimbra service to start or stop were modified to satisfy large (more than 4 minutes) Zimbra start or stop actual times.
  }
\end{itemize}

\section {Zimbra OCF source code}

You can find Zimbra OCF source code at appendix {\ref{cha:zimbra-ocf-source-code} - Zimbra OCF Source Code}.



