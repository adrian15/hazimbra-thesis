% Corosync and Pacemaker installation


\chapter{Corosync and Pacemaker installation}
\label{chap:corosync-pacemaker-installation}
This chapter explains the Corosync and Pacemaker installation.

\section {About Corosync}
The Corosync Cluster Engine is a group communication system for implementing high availability within applications. Among its features we can find:
\begin{itemize}
  \item Create replicated state magchines thanks to a closed process group communication model
  \item Restart application process if it fails thanks to simple availability manager
  \item A configuration and statistics in-memory database
  \item A quorum system that notifies applications when quorum is achieved or lost.
\end{itemize}


The software is designed to operate on UDP/IP and InfiniBand networks natively.

We can find more information about Corosync at its Wikipedia article (\cite{WikipediaCorosync}) and at its webpage (\cite{CorosyncWebpage}).

Corosync enables a group communication system so that Pacemaker can talk to all the cluster nodes without having to implement communication capabilities.

\section {\label{sec:about-pacemaker}About Pacemaker}

Pacemaker is an Open Source, High Availability resource manager suitable for both small and large clusters (\cite{PacemakerWebpage}) which features:
\begin{itemize}
  \item Detection and recovery of machine and application-level failures
  \item Supports practically any redundancy configuration
  \item Supports both quorate and resource-driven clusters
  \item Configurable strategies for dealing with quorum loss (when multiple machines fail)
  \item Supports application startup/shutdown ordering, regardless machine(s) the applications are on
  \item Supports applications that must/must-not run on the same machine
  \item Supports applications which need to be active on multiple machines
  \item Supports applications with multiple modes (eg. master/slave)
  \item Provably correct response to any failure or cluster state. The cluster's response to any stimuli can be tested offline before the condition exists
\end{itemize}

Pacemaker let us manage the high availability cluster as a single system from anyone of the cluster nodes. In order to interact with each of the nodes it needs Corosync communication capabilities.

\section {Corosync installation}
\textbf{In both nodes} we just need to install Corosync packages for Ubuntu 12.04.

We need to issue this command:
\begin{verbatim}
apt-get install corosync
\end{verbatim}


\section {Pacemaker installation}
\textbf{In both nodes} we just need to install Pacemaker packages for Ubuntu 12.04.

We need to issue this command:
\begin{verbatim}
apt-get install pacemaker
\end{verbatim}

We can safely ignore this warning:
\begin{verbatim}
Warning: The home dir /var/lib/heartbeat
 you specified already exists.
Adding system user `hacluster' (UID 105) ...
Adding new user `hacluster' (UID 105) with group `haclient' ...
The home directory `/var/lib/heartbeat' already exists.
 Not copying from `/etc/skel'.
adduser: Warning: The home directory `/var/lib/heartbeat'
 does not belong to the user you are currently creating.
Processing triggers for libc-bin ...
ldconfig deferred processing now taking place
\end{verbatim}
