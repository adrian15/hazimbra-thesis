% thesis.tex
%
% This file is root file for an example thesis written using the
% University of Wisconsin-Madison LaTeX Style file.
%
% It is provided without warranty on an AS IS basis.


%=====================================================================
% Document Style
%=====================================================================
% Choose only one of the following document classes:
%
% for a 12 Point UW PhD Thesis without Margin Check
\documentclass[a4paper,12pt,twoside]{book}
%
% for a 10 Point UW PhD Thesis with Margin Check
%\documentclass[10pt,margincheck]{withesis}
%
% The margincheck option flags lines which overflow their hbox with a black
%  box at the end of the line.  This usually (but not always) indicates a
%  margin violation on the right margin.  Left margin violations aren't
%  indicated and if the margin violation is large enough, there isn't room
%  for the black box to be visiable.  
%
% This option can be also used in conjunction with the msthesis option.
%
% or for a 12 Point UW Masters Thesis
%\documentclass[12pt,msthesis]{withesis}
%
% or for a 10 Point UW Masters Thesis
%\documentclass[10pt,msthesis]{withesis}
%
% The msthesis option changes the page margins from 1" all around
% (the PhD format) to 1.25" left and 1" remaining margins (MS format).
% The defaults for degree and thesis are changed to be MS and thesis.
% These defaults can be overridden if the margins for the MS thesis
% are desired for other documents.

% To include optional packages, use the \usepackage command.
%  The package epsfig is used to bring in the Encapsulated PostScript
%    figures into the document.
%  The package times is used to change the fonts to Times Roman; however
%    because the times typewriter font looks odd, the original LaTeX
%    Computer Modern font is kept for the typewriter font using
%      \renewcommand{\ttdefault}{cmtt}
%    Note that Times Roman is a PostScript font and therefore, the document
%    cannot be correctly viewed from the *.dvi file.  It should be converted
%    to a *.ps file first and then viewed with a PostScript previewer...
\usepackage{epsfig}
\usepackage{graphicx}
\usepackage{url}
\usepackage{times}
\usepackage{datetime}
\renewcommand{\ttdefault}{cmtt}

\begin{document}

\bibliographystyle{plain}

\newcommand{\degree}[1]{\gdef\@degree{#1}}
\newcommand{\masterthesis}{\gdef\@doctype{A Master Thesis}}
\newcommand{\department}[1]{\gdef\@department{(#1)}}

\def\advisortitle#1{\gdef\@advisortitle{#1}}
\def\advisorname#1{\gdef\@advisorname{#1}}
\gdef\@advisortitle{Professor}
\gdef\@advisorname{Default Professor}

%=============================================================================
% COPYRIGHTPAGE
%=============================================================================
% The copyright must do the following:
% - start a new page with no number
% - place the copyright text centered at the bottom.
%=============================================================================
\def\copyrightpage{
  \newpage
  \thispagestyle{empty}    % No page number
  \addtocounter{page}{-1}
  \chapter*{}            % Required for \vfill to work
  \begin{center}
   \vfill
   \copyright\ Copyright by \actualauthor\ \date\\
   All Rights Reserved
  \end{center}}

\newcommand{\twoskip}{1.5}
\newcommand{\doublespace}
  {\renewcommand{\baselinestretch}{\twoskip}\Large\normalsize}

% prelude.tex
%   - titlepage
%   - dedication
%   - acknowledgments
%   - table of contents, list of tables and list of figures
%   - nomenclature
%   - abstract
%============================================================================


\clearpage\pagenumbering{gobble}  % This makes the page numbers Roman (i, ii, etc)



% TITLE PAGE
%   - define \title{} \author{} \date{}
\newcommand{\actualtitle}{Zimbra 8 High Availability on Ubuntu 12.04}
\title{\actualtitle}
\newcommand{\actualauthor}{Adri\'an Gibanel L\'opez}
\author{\actualauthor}
\date{2013}
%   - The default degree is ``Doctor of Philosophy''
%     (unless the document style msthesis is specified
%      and then the default degree is ``Master of Science'')
%     Degree can be changed using the command \degree{}
\degree{M\`aster en Enginyeria de Programari Lliure}
%   - The default is dissertation, unless the document style
%     msthesis was specified in which case it becomes thesis.
%     If msthesis is specified for the MS margins, you can
%     still have a dissertation if you specify \disseration
%\disseration
%   - for a masters project report, specify \project
%\project
%   - for a preliminary report, specify \prelim
%\prelim
%   - for a masters thesis, specify \thesis
\masterthesis
%   - The default department is ``Electrical Engineering''
%     The department can be changed using the command \department{}
\department{Dept. d'Informatica i Enginyeria Industrial}
%   - once the above are defined, use \maketitle to generate the titlepage
\advisorname{Josep Maria Rib\'o Balust}
\advisortitle{Professor}
\newdateformat{udlthesisdate}{\monthname[\THEMONTH] \THEYEAR}

% Title - BEGIN
\let\textquotedbl="
\pagestyle{empty}~

~

~

\begin{center}
{\LARGE \title{}}
\par\end{center}{\LARGE \par}

~
\begin{center}
\includegraphics[scale=0.4]{img/udl}
\par\end{center}
~

~

\begin{center}
{\large Universitat de Lleida}
\par\end{center}{\large \par}

\begin{center}
{\large Escola Polit\`ecnica Superior}
\par\end{center}{\large \par}

\begin{center}
{\large M\`aster en Enginyeria de Programari Lliure}
\par\end{center}{\large \par}

%\begin{center}
%{\large Dept. d'Informatica i Enginyeria Industrial}
%\par\end{center}{\large \par}


~

~
\begin{center}
{\large Treball de final de m\`aster}
\par\end{center}{\large \par}


~
\begin{center}
{\large \textbf{Zimbra 8 High Availability on Ubuntu 12.04}}
\par\end{center}{\large \par}


~

~


\hfill {\large Autor/a: Adri\'an Gibanel L\'opez}
\par{\large \par}

~

\hfill {\large Director/s: Josep Maria Rib\'o Balust}
\par{\large \par}

~

\hfill {\large Setembre 2013}
\par{\large \par}

\newpage

Copyright (c) 2013 Adrian Gibanel Lopez.

Permission is granted to copy, distribute and/or modify this document

under the terms of the GNU Free Documentation License, Version 1.3

or any later version published by the Free Software Foundation;

with no Invariant Sections, no Front-Cover Texts, and no Back-Cover
Texts.

A copy of the license is included in the section entitled \textquotedbl{}GNU

Free Documentation License\textquotedbl{}.

\newpage


% Title - END

% DEDICATION
\newpage
\null\vfil
\begin{center}
To the bTactic crew.
\end{center}
\par\vfil\newpage

% ACKNOWLEDGMENTS

\chapter*{ACKNOWLEDGMENTS}
%\doublespace
I acknowledge Richard M. Stallman for his personal commitment to the free software movement.
\par\newpage




% ABSTRACT

\section* {ABSTRACT}
             \thispagestyle{empty}
                  \addtocounter{page}{-1}
                \begin{center}
                  {\bf\expandafter\uppercase\expandafter{\actualtitle}}\\
                  \vspace{12pt}
                  \actualauthor \\
                  \vspace{12pt}
                  Under the supervision of Professor Josep Maria Rib\'o Balust\\
                  At the Universitat de Lleida
                \end{center}

% abstract.tex
%
% This file has the abstract for the withesis style documentation
%
% Eric Benedict, Aug 2000
%
% It is provided without warranty on an AS IS basis.

\noindent       % Don't indent this paragraph.
The purpose of this master thesis is to design and test the setup of a Zimbra 8 Open Source Edition (OSE) High Availability System (HA) in Ubuntu 12.04.


\vspace*{0.5em}
\noindent       % Don't indent this paragraph.
A HA system has been proposed and tested in a laboratory environment. Its setup has been documented in its all length.

\vspace*{0.5em}
\noindent       % Don't indent this paragraph.
The master thesis shows that thanks to some minor modifications to Zimbra OSE core and thanks to freely available Open Source HA software one can achieve a HA Zimbra OSE system.

\vspace*{0.5em}
\noindent       % Don't indent this paragraph.
The proposed HA Zimbra OSE system can be improved in many ways and the author suggests several ways of doing so.

\pagestyle{headings}

% CONTENTS, TABLES, FIGURES
\tableofcontents
%\listoftables
\listoffigures
\newpage

\clearpage\pagenumbering{arabic} % This makes the page numbers Arabic (1, 2, etc)

% Problem description


\chapter{Introducing Zimbra High Availability}
This thesis was written to reflect the state of the art in High Availability methods for Zimbra Open Source Edition.  

\section {High Availability}
High availability is a system design approach and associated service implementation that ensures a prearranged level of operational performance will be met during a contractual measurement period. 

High Availability, or HA as it is abbreviated, refers to the availability of resources in a computer system, in the wake of component failures in the system. This can be achieved in a variety of ways, spanning the entire spectrum ranging at the one end from solutions that use custom and redundant hardware to ensure availability, to the other end to solutions that provide software solutions using off-the-shelf hardware components. The former class of solutions provide a higher degree of availability, but are significantly more expensive, than the latter class. This has led to the popularity of the latter class, with almost all vendors of computer systems offering various HA products. Typically, these products survive single points of failure in the system.

These HA systems usually ensure a prearranged level of operational performance will be met during a contractual measurement period. 

High availability systems typically operate 24x7 and usually require built in redundancy built-in redundancy to minimize the risk of downtime due to hardware and/or telecommunication failures

Availability can be measured relative to "100\% operational" or "never failing." A widely-held but difficult-to-achieve standard of availability for a system or product is known as "five 9s" (99.999 percent) availability.

TODO: Reference to Wikipedia article.
\begin{verbatim}
http://en.wikipedia.org/wiki/High_availability
\end{verbatim}


TODO: IEEE Task Force on Cluster Computing - February 2011 -
\begin{verbatim}
http://web.archive.org/web/20110216113021/http://www.ieeetfcc.org/high-availability.html
\end{verbatim}

TODO:
\begin{verbatim}
http://www.bcmpedia.org/wiki/High_Availability
\end{verbatim}



\section {Vmware Zimbra OSE}
VMware Zimbra is a complete email, address book, calendar and tasks solution that can be accessed from the Zimbra Web Client, Zimbra Desktop offline client, Outlook and a variety of other standards-based email clients and mobile devices. It can be deployed as a traditional binary install on Linux, or as a software virtual appliance, commonly referred to as Zimbra appliance.

Among the Zimbra Collaboration Server (ZCS) versions this paper will abord the ZCS Open Source Edition also known as Zimbra OSE.

Vmware Zimbra OSE will be refered most of the times as Zimbra.

TODO: http://www.zimbra.com/learn/

\section {Vmware Zimbra OSE High Availability}
Zimbra OSE High Availability is the attempt to confer high availability to each one of the Zimbra Collaboration Server components so that the risk of downtime due to hardware and/or telecommunication failures is minimized. High availability is usually implemented thanks to High Availability software aimed at Gnu/Linux originally which it's adapted to the Zimbra OSE setup.


\section{History}
There have been already some documents that implemented Zimbra High Availability but they were aimed at Zimbra 6 version which it's aimed to work (among others) in Ubuntu 8.04 64 bit. Those instructions were based on Gnu/Linux High Availability software (heartbeat) which no longer is used for High Availability purposes nowadays.

Last year Thesis' author wrote an Spanish howto in order to be able to work with Zimbra 7 and Ubuntu 10.04 64 bit. Finally on September 13th 2012 Vmware announced Zimbra 8 which could be run in Ubuntu 12.04.

\section {Main thesis topic}
This paper aims to show a working implementation of the current state of the art technologies on High Availability in Gnu/Linux applied to the Vmware Zimbra case.

\section {Choosen technologies and solution}
Initially we have decided to use initially the same High Availability technologies as the forementioned howto. Some of these technologies are: Corosync (Configuration files synchronisation), Distributed Replicated Block Device (DRBD, sharing storage between hosts), OCF (Cluster control scripts), and Pacemaker (Cluster management and setup).

In order to validate the current thesis a set of virtual machines will be setup to check that even with Zimbra 8 and Ubuntu 12.04 all the steps in the original instructions can be reproduced. That means that we go through these steps: Setup network in both hosts, setup DRBD, initial Pacemaker setup, corosync installation, pacemaker installation, corosync setup, startup script disabling, DRBD script boot disabling, Pacemaker final setup, Pacemaker case of use tests.


% HA Schema


\chapter{High availability schema}
\label{chap:ha-schema}
This chapter explains the high availability system that will be tested through the Thesis. 

\section {Purpose}
The proposed high availability system is an Active/passive configuration. An Active/passive cluster provides a fully redundant instance of each node, which is only brought online when its associated primary node fails (\cite{HAClusterNodeConfigurations}).

In our case, the primary node will act as the active server and it will provide Zimbra services such as web server, smtp, imap, etc. The secondary node will be idle just waiting for the primary node to fail and bring Zimbra services online when that event happens. In addition the secondary node also mirrors Zimbra data partition in the background thanks to DRBD.

\section {Main schema}
There are two servers which we will be named as the primary one and the secondary one. They are linked by means of two connections: The service and the communication link.

The service link is the main network interface which is connected via a normal switch. It will serve content to the final users. The communication link, which is used for the cluster management and synchronization is done via a crossover cable.

We can see the main schema, where we have added two final clients at figure \ref{fig:main-schema}.

\begin{figure}
  \centering
    \includegraphics[width=0.8\textwidth]{img/ha_main_schema.eps}
  \caption{High Availability main schema}
  \label{fig:main-schema}
\end{figure}

\section {Primary server}
\subsection {Specifications}
The primary server specifications are as follow:
\begin{itemize}
  \item RAM: 4 GB
  \item Hard disk: 100 GB
  \item Processor: 2 x 2,40 Ghz
\end{itemize}

\section {Secondary server}
\subsection {Specifications}
The primary server specifications are as follow:
\begin{itemize}
  \item RAM: 4 GB
  \item Hard disk: 100 GB
  \item Processor: 2 x 2,40 Ghz
\end{itemize}

\section {\label{sec:virtualbox-implementation}Virtualbox implementation}
\subsection {Introduction}
Although in production environments High Availability systems are implemented in Physical servers or highly optimized virtualized servers, we are going to use Oracle VM Virtualbox software to emulate the described system. This section summarizes how to create both virtual machines and link them.

\subsection {\label{subsec:primary-virtual-machine-creation}Primary Virtual Machine creation}
We click on \textit{Machine} menu and select \textit{New} option. The Create Virtual Machine wizard will appear.

\subsubsection {Name and operating system}
\begin{itemize}
  \item Name: PrimaryZimbraHA
  \item Type: Linux
  \item Ubuntu (64 bit)
\end{itemize}

\subsubsection {Memory size}
Zimbra needs: 2048 MB as a minimum.
\subsubsection {Hard drive}
We select \textit{Create a virtual hard drive now}, \textit{Virtualbox Disk Image} as the hard drive file type, Dynamically allocated (so that the hard drive file only uses space as it fills up).

We leave the default File location and select hard disk size as 110 GB which is quite bigger than the strictly needed for our high availability system.

\subsection {\label{subsec:service-link-primary}Service link network on Primary Virtual Machine}
We select \textit{PrimaryZimbraHA} virtual machine and click on \textit{Machine} menu and then in \textit{Settings} option. We will make sure we are in \textit{Network} section.

We will use default \textit{Adapter 1} for service link. We are going to summarize its setup:
\begin{itemize}
  \item Attached to: \textit{Internal Network}
  \item Name: ZimbraHAService
\end{itemize}

Finally we click on OK for saving changes.

\subsubsection {Secondary Virtual Machine creation}
In order to create secondary virtual machine we can either repeat the same steps as in \textbf{\ref{subsec:primary-virtual-machine-creation} Primary Virtual Machine creation}. Or we can make a linked clonation of the original machine. We will describe the latter option.

We select PrimaryZimbraHA virtual machine and then in \textit{Machine} menu we select \textit{Clone} option.

\textbf{New machine name}
\begin{itemize}
  \item New machine name: SecondaryZimbraHA
  \item Reinitialize the MAC address of all network cards: Checked
\end{itemize}

We select \textit{Linked clone} as Clone type.

Finally we click on \textit{Clone} button so that cloning is performed.

\subsection {Service link network on Secondary Virtual Machine}
As we did in subsection \textbf{\ref{subsec:service-link-primary} Service link network on Primary Virtual Machine} we select \textit{SecondaryZimbraHA} virtual machine and click on \textit{Machine} menu and then in \textit{Settings} option. We will make sure we are in \textit{Network} section.

We will use default \textit{Adapter 1} for service link. We are going to summarize its setup:
\begin{itemize}
  \item Attached to: \textit{Internal Network}
  \item Name: \textit{ZimbraHAService}
\end{itemize}

If we have cloned the virtual machine settings might be correct by default.

Finally we click on OK for saving changes.

\subsection {Communication link}
For both PrimaryZimbraHA and SecondaryZimbraHA virtual machines we will perform a very similar operation than the one done in subsection \textbf{\ref{subsec:service-link-primary} Service link network on Primary Virtual Machine}.

But now we make sure that we \textit{Adapter 2} is enabled as an \textit{Internal Network} which name is \textit{ZimbraHACommunication}.

\subsection {NAT link}
In order to make installation easier we will enable \textit{Adapter 3} in both virtual machines so that it can use the host Internet in order to fetch packages and perform post installation setup.

Similarly to subsection \textbf{\ref{subsec:service-link-primary} Service link network on Primary Virtual Machine} we make sure that \textit{Adapter 3} is enabled and that it's attached to NAT.

\subsection {Email client Virtual Machine}
A Virtual Machine whose only purpose is to test high availability from a service link point of view might be added if needed. We're not to cover the installation and its setup here. We will just mention its network setup is similar to PrimaryZimbraHA and SecondaryZimbraHA but removing the second interface which serves for communication link and that, of course, doesn't make sense in an Email client VM.
% Operating System installation


\chapter{Operating System installation}
\label{chap:operating-system-installation}
This chapter explains the Operating System installation.

\section {Introduction}
We will use Ubuntu 12.04 in its 64bit mode because it is one of the official supported Operating System for Zimbra 8 versions. We denote an external DRBD metadata as DRBD-Meta-Disk. We can understand it is an special partition that helps DRBD system to track changes between synchronised partitions between both primary and secondary nodes. We can find a more accurate definition at Linbit site: \cite{LinbitDRBDInternals}.

These instructions are valid for both primary and secondary nodes. The only difference is that each one of them will have a different hostname.

An Ubuntu Gnu/Linux installation might be as complex as of having Logical Volumes Group (LVM) for the ease of hard disk space management. In this installation we will avoid that kind of partitions so that final Pacemaker setup (Chapter \ref{chap:pacemaker-final-setup}) is easier.

\section {Ubuntu 12.04 64 bit minimal}
In order to track all the requisites and just install what the high availability system needs we will use an Ubuntu minimal disk for installation. These disks can be downloaded from \cite{UbuntuMinimalDisk}.

The used download was: \textit{Ubuntu 12.04 "Precise Pangolin" Minimal CD} from the \textit{64-bit PC (amd64, x86\_64)} section. 

\subsection {Installer boot menu}
We just press Return key to select default boot option: \textit{Install}.
\subsection {Select a language}
We choose the desired language: \textit{English}
\subsection {Select your location}
We choose our location: \textit{United States}
\subsection {Configure the keyboard}
We choose \textit{no} for semi automated keyboard detection. That let us choose \textit{English (US} as country of origin for the keyboard. Finally we select default \textit{English (US)} keyboard.
\subsection {Network}
The network detection will be automatically sorted out if one our of interfaces is bridge in a network provided of a DHCP server. Once we boot into the operating system we will setup network manually. We are asked our hostname. We choose: \textit{primary} in case we are installing primary node and \textit{secondary} in case we are installing secondary node.
\subsection {Ubuntu archive mirror country}
We select default \textit{United States} and its associated mirror: \textit{us.archive.ubuntu.com}. When asked for HTTP proxy information we just press Return key to continue.
\subsection {Checking Ubuntu mirror}
Ubuntu installation will perform some background checkings and downloads without the user being noticed. Then the installation suddenly continues by retrieving and installing additional packages and components.
\subsection {Set up users and passwords}
We first are asked to write \textit{Full name for the new user} and then \textit{Username for the account}. We will provide both of them. We are also prompted twice the user password. When asked we choose not to encrypt the home directory.
\subsection {Configure the clock}
Given our detected physical location we just press Return key to validate it.
\subsection {Partition disks}
\subsubsection {Introduction}
Assuming a 1.8 Terabyte hard disk in order to setup DRBD-Meta-Disk there's enough with 59 megabytes. We will be on the safe side and setup it with a 150 megabytes size. In order to safe calculate other DRBD meta disk partitions we can check \cite{LinbitDRBDInternals}.

We will not be using a SWAP partition because Zimbra works better without it. If a SWAP partition would be needed to be used we encourage to add it as a fixed size secondary hard disk.

TODO: According to Zimbra Performance Tunning Guidelines

\subsubsection {Manual partitioning}
We select \textit{Manual} partitioning method. We then select our hard disk in our case: \textit{SCSI1 (0,0,0) (sda) - 107.4 GB ATA VBOX HARD DISK}. When asked to create a new empty partition table on the device we reply \textit{Yes}.
Now we are back at Partition disk screen. For each one of the partitions to be created we should select \textit{FREE SPACE} under our hard disk. Then select create a new partition.  Once a partition has been defined we will confirm its settings by selecting \textit{Done setting up the partition}. We will describe in an schematic manner how the partitions should be created.

\textbf{Root partition}
\begin{itemize}
  \item New partition size: \textit{10 GB}
  \item Type for the new partition: \textit {Primary}
  \item Location for the new partition: \textit{Beginning}
\end{itemize}
Partition settings for Root partition will be left as default except for the Bootable flag which should be set to \textit{on}. These were:
\begin{itemize}
  \item Use as: \textit{Ext4 Journaling file system}
  \item Mount point: \textit{/}
  \item Mount options: \textit{defaults}
  \item Label:	\textit{none}
  \item Reserved blocks: \textit{5\%}
  \item Typical usage: \textit{standard}
  \item Bootable flag: \textbf{\textit{on}}
\end{itemize}


\textbf{DRBD-Meta-Disk partition}
\begin{itemize}
  \item New partition size: \textit{150 MB}
  \item Type for the new partition: \textit {Primary}
  \item Location for the new partition: \textit{Beginning}
\end{itemize}
Partition settings for DRBD-Meta-Disk partition were finally:
\begin{itemize}
  \item Use as: \textit{do not use}
  \item Bootable flag: \textit{off}
\end{itemize}

\textbf{ZimbraData partition}
\begin{itemize}
  \item New partition size: \textit{Rest of disk space}
  \item Type for the new partition: \textit {Primary}
  \item Location for the new partition: \textit{Beginning}
\end{itemize}
Partition settings for ZimbraData partition were finally:
\begin{itemize}
  \item Use as: \textit{Ext4 Journaling file system}
  \item Mount point: \textbf{\textit{/opt}}
  \item Mount options: \textbf{\textit{defaults}}
  \item Label:	\textit{none}
  \item Reserved blocks: \textit{0\%}
  \item Typical usage: \textit{standard}
  \item Bootable flag: \textit{off}
\end{itemize}
We are using \textit{/opt} partition for Zimbra Data because Zimbra programs and its data is stored in \textbf{/opt/zimbra} by default.

Now we are going to confirm all our created partitions by selecting \textit{Finish partitioning and write changes to disk}.
When asked to return to the partitioning menu because a swap partition is needed we will just skip it by saying: \textit{No}. Finally we confirm that we want to write changes to disk by selecting: \textit{Yes}.

\subsection {Configuring x11-common}
We select: \textit{No automatic updates} when asked how to manage upgrades just to make things easier. In a production environment we must select \textit{Install security updates automatically}.

\subsection {Software selection}
When asked which software to install we will only check \textit{OpenSSH server} for installation in order to keep the installation as minimal and functional as possible.

\subsection {Install the GRUB}
When asked to Install teh GRUB boot loader to master boot record we select Yes.

\subsection {Finish the installation}
When we are asked if the system clock is set to UTC we select Yes. Once clock it's been set we need to do a change in Virtualbox. In the running Virtual machine we select \textit{Devices} menu, CD/DVD devices submenu and then we uncheck the minimal Ubuntu 12.04 64 bit iso.

If asked we force unmount in Virtualbox.

Finally we click on \textit{Continue} to finish the installation.

% Network Setup


\chapter{Network setup}
\label{chap:network-setup}
This chapter explains the network setup.

\section {Network schema}
We can just check the High Availability main schema (figure \ref{fig:main-schema}) where network has been already described. There are three networks. The service link is the main network interface for serving content to the final users. The communication link, which it's used for the cluster management and syncronisation is done via a crossover cable. Finally NAT link gives the machines access to Internet.

\section {VirtualBox Network Implementation}

In order to implement this Network schema in VirtualBox the service link will be setup by the first virtual network card in each virtual machine. Both of these virtual network card will be setup in internal mode so that they believe to be connected via a virtual switch. The communication link will be setup by the second virtual network card in each virtual machine connected to a VirtualBox internal network, that means that they will be connected via a virtual switch given by Virtualbox which it's isolated from other networks.

TODO: NAT link

TODO: Add VirtualBox Network figure.


\section {Network setup}

\subsection {Primary server}
\subsubsection {Service link}
Primary server's Service link network setup consists of a Class C configuration where the network card address is 192.168.77.201, as per being a Class C its netmask is 255.255.255.0 and thus its broadcast is 192.168.77.255.

\begin{verbatim}
auto eth0
iface eth0 inet static
        address 192.168.77.201
        netmask 255.255.255.0
        broadcast 192.168.77.255
\end{verbatim}

\subsubsection {Communication link}
Primary server's Communication link network setup consists of a Class C configuration where the network card address is 10.0.2.201, as per being a Class C its netmask is 255.255.255.0 and thus its broadcast is 10.0.2.255. As a gateway it will be using the first network address in the network range which it's 10.0.2.1.

\begin{verbatim}
auto eth0
iface eth0 inet static
        address 10.0.2.201
        netmask 255.255.255.0
        broadcast 10.0.2.255
\end{verbatim}
\subsection {Secondary server}
\subsubsection {Service link}
Secondary server's Service link network setup consists of a Class C configuration where the network card address is 192.168.77.202, as per being a Class C its netmask is 255.255.255.0 and thus its broadcast is 192.168.77.255.

\begin{verbatim}
auto eth0
iface eth0 inet static
        address 192.168.77.202
        netmask 255.255.255.0
        broadcast 192.168.77.255
\end{verbatim}

\subsubsection {Communication link}
Secondary server's Communication link network setup consists of a Class C configuration where the network card address is 10.0.2.202, as per being a Class C its netmask is 255.255.255.0 and thus its broadcast is 10.0.2.255. As a gateway it will be using the first network address in the network range which it's 10.0.2.1.

\begin{verbatim}
auto eth0
iface eth0 inet static
        address 10.0.2.202
        netmask 255.255.255.0
        broadcast 10.0.2.255
\end{verbatim}

TODO: NAT Link

\section {Firewall}

TODO: Describe firewall. Ports that have been left opened.



% Zimbra installation


\chapter{Zimbra installation}
\label{chap:zimbra-installation}
This chapter explains the Zimbra OSE installation.

\section {Zimbra 8.0.4 for Ubuntu 12.04}
Zimbra 8.0.4 for Ubuntu 12.04 in form of a tar.gz file was downloaded from \cite{Zimbra8Download}.
Once downloaded is advised to check its md5sum. Finally we untar it and cd into the untarred directory.

\section {Install script launch}



% DRBD Setup


\chapter{DRBD Setup}
This chapter explains the DRBD Setup.

\section {Introduction}
DRBD is a system that let us mirror a block device via an assigned device. DRBD can be understood as network based raid-1 and it's used in high availability (HA) clusters (\cite{LinbitDRBDWhatIs}).
We're using DRBD to mirror the block device where Zimbra files will be stored on.

\section {Requirements}
We will install DRBD packages for the Ubuntu 12.04 system thanks to the following command:
\begin{verbatim}
sudo apt-get install drbd8-utils
\end{verbatim}
.



\section {DRBD Resource config}
\textbf{To be performed in both servers}.

We will backup main drbd configuration file:
\begin{verbatim}
cp /etc/drbd.conf /etc/drbd.conf.orig
\end{verbatim}
.

We then need to edit:
\begin{verbatim}
/etc/drbd.conf
\end{verbatim}
so that it has:
\begin{verbatim}
resource zimbradata {
  protocol C;
  incon-degr-cmd "halt -f";
  startup {
    degr-wfc-timeout 120; # 2 min
  }
  disk {
    on-io-error detach;
  }
  net {
  }
  syncer {
    rate 10M;
    group 1;
    al-extents 257;
  }
  on zhatest-01.dominio.com {
    device /dev/drbd0;
    disk /dev/zimbra/zimbra;
    address zhatest-01-comm-ip:7788;
    meta-disk /dev/sda2[0];
  }
  on zhatest-02.dominio.com {
    device /dev/drbd0;
    disk /dev/zimbra/zimbra;
    address zhatest-02-comm-ip:7788;
    meta-disk /dev/sda2[0];
  }
}
\end{verbatim}
.

TODO: Explain drbd.conf contents and what they mean more or less.

If we want to modify our drbd.conf to increase sync rate, synchronised devices or any other settings we can check documentation at \cite{LinbitDRBDdrbdconf}.

\section {Start DRBD module}
\textbf{To be performed in both servers}.
\begin{verbatim}
modprobe drbd
\end{verbatim}
.
\section {Metadata disk initialisation}
\textbf{To be performed in both servers}.
We make sure the metadata partition does not have any prior metadata signature.
\begin{verbatim}
dd if=/dev/zero of=/dev/sda2 bs=1K count=100
\end{verbatim}
And we create the zimbradata metadata partition:
\begin{verbatim}
drbdadm create-md zimbradata
\end{verbatim}
.

\section {First DRBD synchronisation}
\textbf{To be performed in both servers}.
\begin{verbatim}
drbdadm up all
\end{verbatim}
where we shouldn't find any incorrect hostname errors.
TODO: Rewrite last line.

We will be asked about usage survey, we just reply that we don't want to participate by saying 'no'.

\textbf{To be performed in Primary server only}.
\begin{verbatim}
drbdadm -- --do-what-I-say primary all
drbdadm -- connect all
\end{verbatim}
.

If we happen to find a \textit{net-config disconnect first} error we can safely ignore it.

In order to check DRBD first synchronisation status we need to check /proc/drbd file with:
\begin{verbatim}
cat /proc/drbd
\end{verbatim}
which will output something like:
\begin{verbatim}
version: 0.7.20 (api:77/proto:74)
SVN Revision: 1743 build by <a href="mailto:phil@mescal">\
phil@mescal</a>, 2005-01-31 12:22:07
0: cs:SyncSource st:Primary/Secondary ld:Consistent
ns:13441632 nr:0 dw:0 dr:13467108 al:0 \
bm:2369 lo:0 pe:23 ua:226 ap:0
[==>..............] sync'ed: 3.1% (7000/7168)M
finish: 1:14:16 speed: 2,644 (2,204) K/sec
1: cs:Unconfigured
\end{verbatim}
.

We must wait for the process to finish in order to continue.







% Zimbra and DRBD Startup Script disabling


\chapter{Zimbra and DRBD startup scripts disabling}
\label{chap:zimbra-drbd-startup-script-disabling}
This chapter explains how to disable both Zimbra and DRBD startup scripts.

\section {Introduction}
We need to disable both default Zimbra and DRBD startup scripts because Pacemaker (see subsection \textbf{\ref{sec:about-pacemaker} About Pacemaker}) will be the responsible for starting and stopping both Zimbra and DRBD thanks to OCF scripts.

The explanation is that it is not safe to start Zimbra at boot because Zimbra needs its filesystem to be mounted. Filesystem needs DRBD to be loaded so that ZimbraData partition exists. All of these requirements are handled by Pacemaker which has been setup for the task. The same reasoning applies to DRBD.

\section {Disable Zimbra startup scripts}
We just have to run \textbf{on both nodes}:
\begin{verbatim}
update-rc.d -f zimbra remove
\end{verbatim}

\section {Disable DRBD startup scripts}
We just have to run \textbf{on both nodes}:
\begin{verbatim}
update-rc.d -f drbd remove
\end{verbatim}

% Corosync setup


\chapter{Corosync setup}
\label{chap:corosync-setup}
This chapter explains the Corosync setup.

\section {About Corosync}
The Corosync Cluster Engine is a group communication system for implementing high availability within applications. Among its features we can find:
\begin{itemize}
  \item Create replicated state machines thanks to a closed process group communication model
  \item Restart application process if it fails thanks to simple availability manager
  \item A configuration and statistics in-memory database
  \item A quorum system that notifies applications when quorum is achieved or lost.
\end{itemize}


The software is designed to operate on UDP/IP and InfiniBand networks natively.

We can find more information about Corosync. Either at its Wikipedia article (\cite{WikipediaCorosync}) or at its web page (\cite{CorosyncWebpage}).

Corosync enables a group communication system so that Pacemaker can talk to all the cluster nodes without having to implement communication capabilities.

\section {Corosync installation}
\textbf{In both nodes} we just need to install Corosync packages for Ubuntu 12.04.

We need to issue this command:
\begin{verbatim}
apt-get install corosync
\end{verbatim}

% COROSYNC Setup

\section {Corosync.conf}

The corosync.conf file on both computers it will be modified to use upnp so that we can use in non multicast environments. In order to use upnp we need to use a corosync version higher than 1.3 but that's not a problem because current version is higher than that.

\subsection {\label{subsec:primary-server-corosync-conf}Primary server Corosync.conf}

We create the file:
\begin{verbatim}
/etc/corosync/corosync.conf
\end{verbatim}
which its contents will be:

\begin{verbatim}
# Please read the openais.conf.5 manual page

totem {
	version: 2

	# How long before declaring a token lost (ms)
	token: 5000

	# How many token retransmits before
	# forming a new configuration
	token_retransmits_before_loss_const: 20

	# How long to wait for join messages
	# in the membership protocol (ms)
	join: 1000

	# How long to wait for consensus to be achieved 
	# before starting a new round of
	# membership configuration (ms)
	consensus: 7500

	# Turn off the virtual synchrony filter
	vsftype: none

	# Number of messages that may be sent by one
	# processor on receipt of the token
	max_messages: 20

	# Limit generated nodeids to 31-bits
	# (positive signed integers)
	clear_node_high_bit: yes

	# Disable encryption
 	secauth: off

	# How many threads to use for encryption/decryption
 	threads: 0

	# Optionally assign a fixed node id (integer)
	 #nodeid: 1234


        rrp_mode: passive


interface {
		member {
			memberaddr: 10.0.66.201
		}
		member {
			memberaddr: 10.0.66.202
		}
        ringnumber: 0
        bindnetaddr: 10.0.66.201
        mcastport: 5405
        }
		transport: udpu
}
amf {
	mode: disabled
}

service {
 	# Load the Pacemaker Cluster Resource Manager
 	ver:       0
 	name:      pacemaker
}

aisexec {
        user:   root
        group:  root
}

logging {
        fileline: off
        to_stderr: yes
        to_logfile: no
        to_syslog: yes
	syslog_facility: daemon
        debug: off
        timestamp: on
   logger_subsys {
           subsys: AMF
           debug: off
tags: enter|leave|trace1|trace2|trace3|trace4|trace6
   }
}
\end{verbatim}

Among other settings the configuration defines that there are two node members which their ips are: 10.0.66.201 and 10.0.66.202. The transport must use unicast protocol (udpu) through 5405 port (and 5404 too). Logging is set to output into syslog and stderr but without debug output. As we are using unicast we need to define the current node ip as the multicast one: 10.0.66.202.

\subsection {Secondary server Corosync.conf}

We create the file:
\begin{verbatim}
/etc/corosync/corosync.conf
\end{verbatim}
which its contents will be:

\begin{verbatim}
# Please read the openais.conf.5 manual page

totem {
	version: 2

	# How long before declaring a token lost (ms)
	token: 5000

	# How many token retransmits before
	# forming a new configuration
	token_retransmits_before_loss_const: 20

	# How long to wait for join messages
	# in the membership protocol (ms)
	join: 1000

	# How long to wait for consensus to be achieved 
	# before starting a new round of
	# membership configuration (ms)
	consensus: 7500

	# Turn off the virtual synchrony filter
	vsftype: none

	# Number of messages that may be sent by one
	# processor on receipt of the token
	max_messages: 20

	# Limit generated nodeids to 31-bits
	# (positive signed integers)
	clear_node_high_bit: yes

	# Disable encryption
 	secauth: off

	# How many threads to use for encryption/decryption
 	threads: 0

	# Optionally assign a fixed node id (integer)
	 #nodeid: 1234


        rrp_mode: passive


interface {
		member {
			memberaddr: 10.0.66.201
		}
		member {
			memberaddr: 10.0.66.202
		}
        ringnumber: 0
        bindnetaddr: 10.0.66.202
        mcastport: 5405
        }
		transport: udpu
}
amf {
	mode: disabled
}

service {
 	# Load the Pacemaker Cluster Resource Manager
 	ver:       0
 	name:      pacemaker
}

aisexec {
        user:   root
        group:  root
}

logging {
        fileline: off
        to_stderr: yes
        to_logfile: no
        to_syslog: yes
	syslog_facility: daemon
        debug: off
        timestamp: on
   logger_subsys {
           subsys: AMF
           debug: off
tags: enter|leave|trace1|trace2|trace3|trace4|trace6
   }
}
\end{verbatim}

You can check subsection \textbf{\ref{subsec:primary-server-corosync-conf} Primary server Corosync.conf} for configuration file settings explanation.

\section {Corosync's Authkey}
In \textbf{primary node} we will create the file:
\begin{verbatim}
/etc/corosync/authkey
\end{verbatim}
thanks to running:
\begin{verbatim}
corosync-keygen
\end{verbatim}
We might be requested to press keys on our keyboard to generate entropy.

Once the file created we will copy the very same file to the \textbf{secondary node} in the same path as in primary server.

In order to secure it in secondary node we will need to run:
\begin{verbatim}
chown root:root /etc/corosync/authkey
chmod 400 /etc/corosync/authkey
\end{verbatim}

\section {Cfgtool}

In \textbf{both nodes} we need to edit:
\begin{verbatim}
/etc/rc.local
\end{verbatim}
in order to add the line:
\begin{verbatim}
/usr/sbin/corosync-cfgtool -r
\end{verbatim}
just before the:
\begin{verbatim}
exit 0
\end{verbatim}
line.

This way we make sure that redudant ring state is reset cluster wide after a fault to re-enable redundant ring.

\section {Corosync startup enabling}
In \textbf{both nodes} in order to enable Corosync at boot we need to edit:
\begin{verbatim}
/etc/default/corosync
\end{verbatim}
file so that:
\begin{verbatim}
START=no
\end{verbatim}
line becomes:
\begin{verbatim}
START=yes
\end{verbatim}
.

\section {Corosync reboot and check}
In \textbf{both nodes} in order to check Corosync startup we need to reboot the machine thanks to:
\begin{verbatim}
sync
shutdown -r now
\end{verbatim}
.

Once the machine has rebooted we can cluster status thanks to:
\begin{verbatim}
crm_mon
\end{verbatim}

In order to check that everything is ok we need to make sure that the output shown in both nodes is the same one.
If both nodes are shown inside Online line that means that both nodes are detected to be online from the Pacemaker point of view.

Here there is a crm\_mon output where both nodes are online:
\begin{verbatim}
============
Last updated: Sun Sep  8 13:06:11 2013
Last change: Sun Sep  8 13:04:19 2013 via crmd on primary
Stack: openais
Current DC: primary - partition with quorum
Version: 1.1.6-9971ebba4494012a93c03b40a2c58ec0eb60f50c
2 Nodes configured, 2 expected votes
0 Resources configured.
============

Online: [ secondary primary ]
\end{verbatim}
% Corosync and Pacemaker installation

\chapter {Pacemaker setup}
\label{chap:pacemaker-setup}

This chapter explains the Pacemaker installation and setup.

\section {\label{sec:about-pacemaker}About Pacemaker}

Pacemaker is an Open Source, High Availability resource manager suitable for both small and large clusters (\cite{PacemakerWebpage}) which features:
\begin{itemize}
  \item Detection and recovery of machine and application-level failures
  \item Supports practically any redundancy configuration
  \item Supports both quorate and resource-driven clusters
  \item Configurable strategies for dealing with quorum loss (when multiple machines fail)
  \item Supports application startup/shutdown ordering, regardless machine(s) the applications are on
  \item Supports applications that must/must-not run on the same machine
  \item Supports applications which need to be active on multiple machines
  \item Supports applications with multiple modes (eg. master/slave)
  \item Provably correct response to any failure or cluster state. The cluster's response to any stimuli can be tested off line before the condition exists
\end{itemize}

Pacemaker let us manage the HA cluster as a single system from anyone of the cluster nodes. In order to interact with each one of the nodes it needs Corosync communication capabilities.


\section {Pacemaker installation}
\textbf{In both nodes} we just need to install Pacemaker packages for Ubuntu 12.04.

We need to issue this command:
\begin{verbatim}
apt-get install pacemaker
\end{verbatim}

We can safely ignore this warning:
\begin{verbatim}
Warning: The home dir /var/lib/heartbeat
 you specified already exists.
Adding system user `hacluster' (UID 105) ...
Adding new user `hacluster' (UID 105) with group `haclient' ...
The home directory `/var/lib/heartbeat' already exists.
 Not copying from `/etc/skel'.
adduser: Warning: The home directory `/var/lib/heartbeat'
 does not belong to the user you are currently creating.
Processing triggers for libc-bin ...
ldconfig deferred processing now taking place
\end{verbatim}
% Initial Pacemaker Setup

\section {Pacemaker reboot and check}
In \textbf{both nodes} in order to check Pacemaker startup we need to reboot the machine thanks to:
\begin{verbatim}
sync
shutdown -r now
\end{verbatim}
.

Once the machine has rebooted we can cluster status thanks to:
\begin{verbatim}
crm_mon
\end{verbatim}

In order to check that everything is ok we need to make sure that the output shown in both nodes is the same one.
If both nodes are shown inside Online line that means that both nodes are detected to be online from the Pacemaker point of view.

Here there is a crm\_mon output where both nodes are online:
\begin{verbatim}
============
Last updated: Sun Sep  8 13:06:11 2013
Last change: Sun Sep  8 13:04:19 2013 via crmd on primary
Stack: openais
Current DC: primary - partition with quorum
Version: 1.1.6-9971ebba4494012a93c03b40a2c58ec0eb60f50c
2 Nodes configured, 2 expected votes
0 Resources configured.
============

Online: [ secondary primary ]
\end{verbatim}

\section {bTactic Zimbra OCF installation}
\textbf{In both nodes} we need to obtain the Zimbra OCF script. Zimbra OCF script is found inside BtacticOCF tar.gz file (\cite{BtacticOCF}) which can be downloaded from BtacticOCF  tar.gz file (\cite{BtacticOrg}).

From the tar.gz we will use the zimbra script which we will copy into:
\begin{verbatim}
/usr/lib/ocf/resource.d/btactic
\end{verbatim}

We will use a temporary directory in order to use it:
\begin{verbatim}
mkdir /tmp/temp
cd /tmp/temp
\end{verbatim}
We download and extract it:
\begin{verbatim}
wget "http://www.btactic.org/btactic_ocf_0.0.2.tar.gz"
tar xzf btactic_ocf_0.0.2.tar.gz
\end{verbatim}
Make the btactic resource directory and copy zimbra file in there:
\begin{verbatim}
mkdir --parents /usr/lib/ocf/resource.d/btactic
cp zimbra /usr/lib/ocf/resource.d/btactic
\end{verbatim}
We finally make sure to give the script executable permissions:
\begin{verbatim}
chmod +x /usr/lib/ocf/resource.d/btactic/*
\end{verbatim}

% Pacemaker Final Setup

\section {\label{sec:pacemaker-final-setup}Pacemaker final setup}
As explained in \textit{section \ref{sec:about-pacemaker} About Pacemaker} Pacemaker is High Availability resource manager, here we will explain how our setup pretends to manage our two servers resources. These instructions should only be performed on \textbf{primary} node.

We need to introduce resource stickiness concept. Resource stickiness controls how much a service prefers to stay running where it is. You may like to think of it as the \textit{cost} of any downtime. By default, Pacemaker assumes there is zero cost associated with moving resources and will do so to achieve \textit{optimal} resource placement (\cite{ClustersFromScratch}).

These are the main settings we define in our configuration:
\begin{itemize}
  \item Setup deletes prior configuration.
  \item DRBD, Filesystem mount and Zimbra Server are setup to work in the same server as they work as in a team.
  \item System stickiness is changed so that Zimbra Server resource is not moved from where it is running to avoid Zimbra unnecessary downtimes.
  \item System are forced to be started in the right order. The right order is: DRBD, Filesystem mount, and Zimbra Server.
  \item We disable stonith (definition on subsection {\ref{subsec:fencing} Fencing}) in order to simplify the setup.
  \item Primary server will be the preferred server where resources need to be run.
\end{itemize}

We make sure that \textit{/tmp/zimbrapacemaker.config} file contents are:
\begin{verbatim}
configure
erase
node primary
node secondary
# Activate failover
property no-quorum-policy=ignore
# Disable stonith
property stonith-enabled=false
# Resource stickiness
rsc_defaults resource-stickiness=100
# Public ip fail over check
primitive ClusterIP ocf:heartbeat:IPaddr2 \
params nic=eth0 ip=192.168.77.203 \
cidr_netmask=24 \
broadcast=192.168.77.255 \
op monitor interval=30s
# Configure zimbra resource
primitive ZimbraServer ocf:btactic:zimbra op \
monitor interval=2min timeout="40s" \
op start interval="0" timeout="360s" \
op stop interval="0" timeout="360s"
# Prefered location: primary node

# Define DRBD ZimbraData
primitive ZimbraData ocf:linbit:drbd params \
drbd_resource=zimbradata op monitor \
role=Master interval=60s op monitor \
role=Slave interval=50s \
op start role=Master interval="0" timeout="240" \
op start role=Slave interval="0" timeout="240" \
op stop role=Master interval="0" timeout="100" \
op stop role=Slave interval="0" timeout="100"
# Define DRBD ZimbraData Clone
ms ZimbraDataClone ZimbraData meta \
master-max=1 master-node-max=1 \
clone-max=2 clone-node-max=1 notify=true
# Define ZimbraFS so that zimbra can use it
primitive ZimbraFS ocf:heartbeat:Filesystem \
params device="/dev/drbd/by-res/zimbradata" \
directory="/opt" fstype="ext4" \
op start interval="0" timeout="60s" \
op stop interval="0" timeout="60s"
group MyZimbra ZimbraFS ZimbraServer

# Everything in the same host
colocation ClusterIP-with-ZimbraDataClone-Master \
inf: ZimbraDataClone:Master ClusterIP

colocation ZimbraDataClone-Master-with-ZimbraFS \
inf: ZimbraFS ZimbraDataClone:Master

colocation ZimbraFS-with-ZimbraServer \
inf: ZimbraServer ZimbraFS


# Everything ordered
order ZimbraDataClone-promote-on-ClusterIP \
inf: \
ClusterIP:start \
ZimbraDataClone:promote

order ZimbraFS-on-ZimbraDataClone-promote \
inf: \
ZimbraDataClone:promote ZimbraFS:start

order ZimbraServer-on-ZimbraFS \
inf: \
ZimbraFS:start ZimbraServer:start

# Prefer primary location
location ClusterIP-prefer-primary-node \
ClusterIP 50: primary
location ZimbraFS-prefer-primary-node \
ZimbraFS 50: primary
location ZimbraServer-prefer-primary-node \
ZimbraServer 50: primary

commit
\end{verbatim}

In order to apply the configuration we will run:
\begin{verbatim}
crm < /tmp/zimbrapacemaker.config
\end{verbatim}

Pacemaker configuration is not trivial. The main document from which the configuration file was adapted and written was Clusters from Scratch (\cite{ClustersFromScratch}).
% High Availability System Management


\chapter{High Availability System Management}
\label{chap:ha-system-management}
This chapter describes some common management tasks that can be used in High Availability systems like ours.

\section {Introduction}
Most of these examples have been adapted from Clusters From Scratch document (\cite{ClustersFromScratch}).

At Zimbra Forums Zimbra on Pacemaker + DRBD howto thread (\cite{ZForumsTaer}) we can find a Zimbra High Availability Howto (\cite{TaerHowtoHAZimbra8}) where we can find more every-day uses of Pacemaker examples.

The reason why we need these examples is that instead of managing a single server installation Zimbra server system now we need to manage a high availability system initially setup by Pacemaker. Some of the most useful management tasks will be described.

\section {DRBD Split Brain recovery}

Split brain is a situation where, due to temporary failure of all network links between cluster nodes, and possibly due to intervention by a cluster management software or human error, both nodes switched to the primary role while disconnected. This is a potentially harmful state, as it implies that modifications to the data might have been made on either node, without having been replicated to the peer. Thus, it is likely in this situation that two diverging sets of data have been created, which cannot be trivially merged (\cite{DrbdSplitBrain}).

This is an example of \textit{cat /proc/drbd} output:
\begin{verbatim}
version: 8.3.11 (api:88/proto:86-96)
srcversion: 93CE421BB73A731BDC72D8E 
 0: cs:WFConnection ro:Primary/Unknown
    ds:UpToDate/DUnknown C r-----
    ns:0 nr:0 dw:0 dr:1977 al:0
      bm:0 lo:0 pe:0 ua:0 ap:0
        ep:1 wo:f oos:153220
\end{verbatim}
where there is an split brain.

In order to fix it we can decide that primary node contents will prevail and that secondary node contents will be discarded. First of all we will need to stop the cluster on \textbf{both nodes} thanks to:
\begin{verbatim}
service corosync stop
\end{verbatim}
.

Then we need to start drbd service manually in \textbf{both nodes} thanks to:
\begin{verbatim}
service drbd start
\end{verbatim}

First of all \textbf{in secondary node} we will discard its data with:
\begin{verbatim}
drbdadm secondary zimbradata
drbdadm -- --discard-my-data connect zimbradata
\end{verbatim}
Then we will need to run \textbf{in primary node}:
\begin{verbatim}
drbdadm connect zimbradata
\end{verbatim}

Once we check that \textit{cat /proc/drbd} has a non split brain state like:
\begin{verbatim}
version: 8.3.11 (api:88/proto:86-96)
srcversion: 93CE421BB73A731BDC72D8E 
 0: cs:Connected ro:Primary/Secondary
   ds:UpToDate/UpToDate C r-----
    ns:5691029 nr:0 dw:0 dr:6002290
      al:0 bm:367 lo:0 pe:0 ua:0
        ap:0 ep:1 wo:f oos:0
\end{verbatim}
we can restart the cluster with running (\textbf{in both nodes}):
\begin{verbatim}
service drbd stop
service corosync start
\end{verbatim}
so that drbd is not handled manually and the cluster takes care of it.

\section {Host down simulation}
These commands simulate that a host is down by stopping both pacemaker and corosync services. We will run them only in the primary server. The secondary server is supposed to take control of the cluster system and start Zimbra.

\begin{verbatim}
service pacemaker stop
service corosync stop
\end{verbatim}

\section {Node recover}
In order to simulate a node recover we will start both pacemaker and corosync services in primary server:
\begin{verbatim}
service corosync start
service pacemaker start
\end{verbatim}

As per our our cluster system stickiness Zimbra service will stay in secondary server.

\section {Resources check}
In order to check the overall Cluster status we just run the cluster management monitor command:
\begin{verbatim}
crm_mon
\end{verbatim}

One example of crm\_mon output where the cluster has not problems at all is:
\begin{verbatim}
============
Last updated: Sun Sep  8 23:28:38 2013
Last change: Sun Sep  8 23:24:33 2013 via cibadmin on primary
Stack: openais
Current DC: secondary - partition with quorum
Version: 1.1.6-9971ebba4494012a93c03b40a2c58ec0eb60f50c
2 Nodes configured, 2 expected votes
5 Resources configured.
============

Online: [ secondary primary ]

 Master/Slave Set: ZimbraDataClone [ZimbraData]
     Masters: [ primary ]
     Slaves: [ secondary ]
 Resource Group: MySystem
     ClusterIP  (ocf::heartbeat:IPaddr2):       Started primary
 Resource Group: MyZimbra
     ZimbraFS   (ocf::heartbeat:Filesystem):    Started primary
     ZimbraServer       (ocf::btactic:zimbra):  Started primary
\end{verbatim}

\section {Move cluster resources temporarily}
In the case that we want to move \textit{temporarily} resources to primary server we should run:
\begin{verbatim}
crm resource move ZimbraServer primary
\end{verbatim}

\section {Revert cluster resources movement}
If we want to return Resource control to the cluster we can run:
\begin{verbatim}
crm resource unmove primary
\end{verbatim}
In our setup, thanks to our defined stickiness the cluster won't perform any resource movement.

\section {Migration testing}
We can simulate the migration by declaring a node in standby. This way the standby node hardware can be fixed. We just have to run:
\begin{verbatim}
crm node standby
\end{verbatim}
in the affected node.

In order to check the migration status we can run:
\begin{verbatim}
crm_mon
\end{verbatim}
.

This is a crm\_mon output while node is migrating from primary node to secondary node:
\begin{verbatim}
============
Last updated: Sun Sep  8 23:33:41 2013
Last change: Sun Sep  8 23:31:50 2013
            via crm_attribute on primary
Stack: openais
Current DC: secondary - partition with quorum
Version: 1.1.6-9971ebba4494012
              a93c03b40a2c58ec0eb60f50c
2 Nodes configured, 2 expected votes
5 Resources configured.
============

Node primary: standby
Online: [ secondary ]

 Master/Slave Set: ZimbraDataClone [ZimbraData]
     Masters: [ secondary ]
     Stopped: [ ZimbraData:1 ]
 Resource Group: MySystem
     ClusterIP  (ocf::heartbeat:IPaddr2):
                            Started secondary
 Resource Group: MyZimbra
     ZimbraFS   (ocf::heartbeat:Filesystem):
                            Started secondary
     ZimbraServer       (ocf::btactic:zimbra):
                            Stopped
\end{verbatim}
as we can see both ClusterIP and ZimbraFS have already started on secondary node and ZimbraServer is not started in secondary node yet.


Finally once we have fixed the hardware we can declare the node online again thanks to:
\begin{verbatim}
crm node online
\end{verbatim}
.

Once again in our setup, thanks to our defined stickiness the cluster won't perform any resource movement.

\section {Starting and stopping resources}
Sometimes, mainly for debugging purposes, is needed to start or stop cluster resources manually.

E.g. in order to start ZimbraFS resource we will issue:
\begin{verbatim}
crm resource start ZimbraFS
\end{verbatim}
In order to stop the same resource we will issue:
\begin{verbatim}
crm resource stop ZimbraFS
\end{verbatim}


% Conclusions and future work


\chapter{Conclusions and future work}
\label{chap:conclusions-and-future-work}
This chapter draws some conclusions and describes some of the improvements that can be applied to the described High Availability System. These improvements can be used in further research.

TODO: The described. Find a better rewording.

\section {Conclusions}

TODO: Objective
TODO: Subjective

In 

{prelude}
{problem-description}
{ha-schema}

we have seen that Zimbra OSE could be improved by the means of a High Availability system.

In 

{operating-system-installation}
{network-setup}
{zimbra-installation}
{drbd-setup}
{zimbra-drbd-startup-script-disabling}
{corosync-setup}
{pacemaker-setup}
{ha-system-management}

we have seen a practical example on how to achieve this High Availability system.

\section {Future work}

\subsection {OVH Datacentre network handling}
There have been some efforts from the bTactic team to handle OVH Datacentre networking thanks to three OCF scripts named: 
ClusterOVHFailover, ClusterHostRoute and ClusterDefaultRoute. These scripts make sense in setup that doesn't use Virtual Rack + RIPE but just normal servers connected via Internet only.

ClusterOVHFailover makes sure an OVH ip-fail-over is failed over from one machine to the another one. This way Zimbra Server is being served by the correct host.
ClusterHostRoute and ClusterDefaultRoute are meant to help to setup OVH networking for ip-fail-over which is not covered by standard network setup found in  default Pacemaker package.

These scripts can be found inside BtacticOCF tar.gz file (\cite{BtacticOCF}) which can be downloaded from BtacticOCF  tar.gz file (\cite{BtacticOrg}).

\subsection {Fencing}
Fencing is the process of locking resources away from a node whose status is uncertain (\cite{LinuxHAFencing}). The default method of fencing in Pacemaker is using stonith. Stonith is a technique for node fencing, that means that the node (either primary or secondary server in our example) that it's supposed to have failed is \textit{shot in head} (\cite{LinuxHAStonith}). That makes sure that the node is actually dead.

In our described example we have disabled stonith. We can improve it by enabling it and using one of the available methods.

Once again bTactic team has developed an stonith script (or fence agent as known per Pacemaker) to make sure an OVH server doesn't use a shared resource. The failing node is rebooted into a Rescue mode which is quite similar to booting a computer with a live CD that doesn't make any change to local hard disks. Fence agent name is: fence\_ovh.

This script is available as a Red Hat's fence-agents package since fence-agents package 4.0.2 release version (\cite{LinuxClusterML201307}).

\subsection {Mysql HA}
One of the Zimbra components is a Mysql database. This Mysql database can be excluded from DRBD syncing and can be setup as an active / active cluster.

This setup will offer a better handling of node failing because you would only loose last mysql queries. In the DRBD case you loose all the file changes that have not been stored into the files that compose actual Mysql database.

The only drawback for this improvement is that Zimbra OSE upgrades tend to be much more difficult when we want to preserve high availability.

\subsection {Project Always ON}

TODO: Project Always ON . http://blog.zimbra.com/blog/archives/2013/09/project-always-on.html
     
%\include{figs}
%\include{bibs}

%\bibliography{refs}              % Make the bibliography

\begin{appendix}
\include{gfdl}
%\include{code}                   % Including computer code listings
%\include{bibref}                 % a BibTeX reference
%\include{math}                   % Complex Equations from the UW Math Department
%\include{acro}                   % A discussion on generating PDF files.
\end{appendix}

\begin{thebibliography}{1}

% problem-description

\bibitem[WIHA]{WikipediaHA} High Availability Wikipedia Article
\url{http://en.wikipedia.org/wiki/High_availability}. Last access 2013-08-20.

\bibitem[ATFC]{TaskForceHA} IEEE Task Force on Cluster Computing. archive.org capture from February 2011.
\url{http://web.archive.org/web/20110216113021/http://www.ieeetfcc.org/high-availability.html}. Last access 2013-08-20.

\bibitem[BCHA]{BCMHA} High Availability Bcmpedia definition
\url{http://www.bcmpedia.org/wiki/High_Availability}. Last access 2013-08-20.

\bibitem[ZLEA]{ZimbraLearn} Zimbra - Learn
\url{http://www.zimbra.com/learn/}. Last access 2013-08-20.

\bibitem[BZHA]{BtacticZimbraHAHowto} ZimbraOSE-Alta disponibilidad con DRBD con una ip-fail-over - 26 de agosto de 2012
\url{http://www.btactic.org/zimbra_ose_alta_disponibilidad_drbd_1_ipfailover.pdf}. Last access 2013-08-20

\bibitem[VWZ8]{VmwareZimbra8Announce} Zimbra Blog - Announcing The General Availability of Zimbra 8
\url{http://blog.zimbra.com/blog/archives/2012/09/announcing-the-general-availability-of-zimbra-8.html}. Last access 2013-08-20.

\end{thebibliography}



%\include{vita}                  % Optional Vita, use \begin{vita} vita text \end{vita}
\end{document}
