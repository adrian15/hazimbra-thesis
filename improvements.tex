% Improvements


\chapter{Improvements}
This chapter describes some of the improvements that can be applied to the described High Availability System. These improvements can be used in further research.

TODO: The described. Find a better rewording.

\section {OVH Datacentre network handling}
There have been some efforts from the bTactic team to handle OVH Datacentre networking thanks to three OCF scripts named: 
ClusterOVHFailover, ClusterHostRoute and ClusterDefaultRoute. These scripts make sense in setup that doesn't use Virtual Rack + RIPE but just normal servers connected via Internet only.

ClusterOVHFailover makes sure an OVH ip-fail-over is failed over from one machine to the another one. This way Zimbra Server is being served by the correct host.
ClusterHostRoute and ClusterDefaultRoute are meant to help to setup OVH networking for ip-fail-over which it's not covered by standard network setup found in  default Pacemaker package.

TODO: Add reference to original howto.

\section {Fencing}
Fencing is the process of locking resources away from a node whose status is uncertain. The default method of fencing in Pacemaker is using stonith. Stonith is a technique for NodeFencing, that means that the node (either primary or secondary server in our example) that it's supposed to have failed is \textit{shot in head}. That makes sure that the node is actually dead.

In our described example we have disabled stonith. We can improve it by enabling it and using one of the available methods.

Once again bTactic team has developed an stonith script (or fence agent as known per Pacemaker) to make sure an OVH server doesn't use a shared resource. The failing node is rebooted into a Rescue mode which it's quite similar to booting a computer with a live cd that doesn't make any change to local hard disks. Fence agent name is: fence\_ovh.

This script is available as a Red Hat's fence-agents package since fence-agents package 4.0.2 release version.


TODO: Reference: 
\begin{verbatim}
http://linux-ha.org/wiki/Fencing
\end{verbatim}

TODO: Reference:
\begin{verbatim}
http://linux-ha.org/wiki/STONITH
\end{verbatim}

TODO: Reference:
\begin{verbatim}
http://www.redhat.com/archives\
/linux-cluster/2013-July/msg00028.html
\end{verbatim}

