% Initial Pacemaker Setup


\chapter{Initial Pacemaker Setup}
\label{chap:initial-pacemaker-setup}
This chapter is about the Pacemaker initial setup that it's needed so that Zimbra can be treated as a resource.

\section {About OCF Resource agents}
A resource agent is a standardized interface for a cluster resource. In translates a standard set of operations into steps specific to the resource or application, and interprets their results as success or failure(\cite{ResourceAgentsWiki}). An OCF resource agent is based on the Open Clustering Framework Resource Agent API specifications(\cite{OCFResourceAgentsWiki}). 

\section {bTactic Zimbra OCF installation}
We need to obtain the Zimbra OCF script. Zimbra OCF script is found inside BtacticOCF tar.gz file (\cite{BtacticOCF}) which can be downloaded from BtacticOCF  tar.gz file (\cite{BtacticOrg}).

From the tar.gz we will use the zimbra script which we will copy into:
\begin{verbatim}
/usr/lib/ocf/resource.d/btactic
\end{verbatim}

We will use a temporary directory in order to use it:
\begin{verbatim}
mkdir /tmp/temp
cd /tmp/temp
\end{verbatim}
We download and extract it:
\begin{verbatim}
wget "http://www.btactic.org/btactic_ocf_0.0.2.tar.gz"
tar xzf btactic_ocf_0.0.2.tar.gz
\end{verbatim}
Make the btactic resource directory and copy zimbra file in there:
\begin{verbatim}
mkdir --parents /usr/lib/ocf/resource.d/btactic
cp zimbra /usr/lib/ocf/resource.d/btactic
\end{verbatim}
We finally make sure to give the script executable permissions:
\begin{verbatim}
chmod +x /usr/lib/ocf/resource.d/btactic/*
\end{verbatim}

