% Problem description


\chapter{Introducing Zimbra High Availability}
This thesis was written to reflect the state of the art in High Availability methods for Zimbra Open Source Edition.  

\section {High Availability}
High availability is a system design approach and associated service implementation that ensures that a prearranged level of operational performance will be met during a contractual measurement period.

TODO: ADD AN EXAMPLE

High Availability, or HA as it is abbreviated, refers to the availability of resources in a computer system, in the event of component failures in the system. This can be achieved in a variety of ways, either with custom and redundant hardware to ensure availability or with software solutions using off-the-shelf hardware components.

The former class of solutions provide a higher degree of availability, but are significantly more expensive than the class. This has led to the popularity of the latter class, with almost all vendors of computer systems offering various HA products. Typically, these products endure single points of failure in the system. (\cite{TaskForceHA})

TODO: ADD AN EXAMPLE

These HA systems usually ensure a prearranged level of operational performance will be met during a contractual measurement period. (\cite{WikipediaHA})

High availability systems typically operate 24x7 and usually require built in redundancy built-in redundancy to minimize the risk of downtime due to hardware and/or telecommunication failures. 

Availability can be measured relative to "100\% operational" or "never failing." A widely-held but difficult-to-achieve standard of availability for a system or product is known as "five 9s" (99.999 percent) availability. (\cite{BCMHA})


\section {Vmware Zimbra OSE}
VMware Zimbra is a complete email, address book, calendar and tasks solution that can be accessed from the Zimbra Web Client, Zimbra Desktop offline client, Outlook and a variety of other standards-based email clients and mobile devices. It can be deployed as a traditional binary install on Linux, or as a software virtual appliance, commonly referred to as Zimbra appliance.

Among the Zimbra Collaboration Server (ZCS) versions this paper will abord the ZCS Open Source Edition also known as Zimbra OSE.

Vmware Zimbra OSE will be refered most of the times as Zimbra.

TODO: http://www.zimbra.com/learn/

\section {Vmware Zimbra OSE High Availability}
Zimbra OSE High Availability is a project which attempts to attain high availability to each one of the Zimbra Collaboration Server components so that the risk of downtime due to hardware and/or telecommunication failures is minimized. High availability is usually implemented using High Availability software aimed at Gnu/Linux originally which it's adapted to the Zimbra OSE setup.


\section{History}

Prior documentation about Zimbra High Availability was written with Zimbra 6 version in mind which it's devoted to work (among others) in Ubuntu 8.04 64 bit. Those instructions were based on Gnu/Linux High Availability software (heartbeat) which no longer is used for High Availability purposes nowadays.


TODO: REF (Last year Thesis' author)

 wrote an Spanish howto in order to be able to work with Zimbra 7 and Ubuntu 10.04 64 bit. Finally on September 13th 2012 Vmware announced Zimbra 8 which could be run in Ubuntu 12.04.

\section {Main thesis topic}
The goal of this Master Thesis is to show a working implementation of the current state of the art technologies on High Availability in Gnu/Linux applied to the Vmware Zimbra case.

\section {Chosen technologies and solution}
Initially, we have decided to use the same High Availability technologies as the forementioned howto. Some of these technologies are: Corosync (Configuration files synchronisation), Distributed Replicated Block Device (DRBD, sharing storage between hosts), OCF (Cluster control scripts), and Pacemaker (Cluster management and setup).

TODO: Explicar de d\'onde sale Corosync, qu\'e hace y por qu\'e aparece aqu\'i.

A set of virtual machines will be setup to check that even with Zimbra 8 and Ubuntu 12.04 all the steps in the original instructions can be reproduced. These steps are: Setup network in both hosts, setup DRBD, initial Pacemaker setup, corosync installation, pacemaker installation, corosync setup, startup script disabling, DRBD script boot disabling, Pacemaker final setup, Pacemaker case of use tests.

\section {Structure of the document}
TODO


