% Corosync and Pacemaker installation

\chapter {Pacemaker setup}
\label{chap:pacemaker-setup}

This chapter explains the Pacemaker installation and setup.

\section {\label{sec:about-pacemaker}About Pacemaker}

Pacemaker is an Open Source, High Availability resource manager suitable for both small and large clusters (\cite{PacemakerWebpage}) which features:
\begin{itemize}
  \item Detection and recovery of machine and application-level failures
  \item Supports practically any redundancy configuration
  \item Supports both quorate and resource-driven clusters
  \item Configurable strategies for dealing with quorum loss (when multiple machines fail)
  \item Supports application startup/shutdown ordering, regardless machine(s) the applications are on
  \item Supports applications that must/must-not run on the same machine
  \item Supports applications which need to be active on multiple machines
  \item Supports applications with multiple modes (eg. master/slave)
  \item Provably correct response to any failure or cluster state. The cluster's response to any stimuli can be tested off line before the condition exists
\end{itemize}

Pacemaker let us manage the HA cluster as a single system from anyone of the cluster nodes. In order to interact with each one of the nodes it needs Corosync communication capabilities.


\section {Pacemaker installation}
\textbf{In both nodes} we just need to install Pacemaker packages for Ubuntu 12.04.

We need to issue this command:
\begin{verbatim}
apt-get install pacemaker
\end{verbatim}

We can safely ignore this warning:
\begin{verbatim}
Warning: The home dir /var/lib/heartbeat
 you specified already exists.
Adding system user `hacluster' (UID 105) ...
Adding new user `hacluster' (UID 105) with group `haclient' ...
The home directory `/var/lib/heartbeat' already exists.
 Not copying from `/etc/skel'.
adduser: Warning: The home directory `/var/lib/heartbeat'
 does not belong to the user you are currently creating.
Processing triggers for libc-bin ...
ldconfig deferred processing now taking place
\end{verbatim}
% Initial Pacemaker Setup

\section {Pacemaker reboot and check}
In \textbf{both nodes} in order to check Pacemaker startup we need to reboot the machine thanks to:
\begin{verbatim}
sync
shutdown -r now
\end{verbatim}
.

Once the machine has rebooted we can cluster status thanks to:
\begin{verbatim}
crm_mon
\end{verbatim}

In order to check that everything is ok we need to make sure that the output shown in both nodes is the same one.
If both nodes are shown inside Online line that means that both nodes are detected to be online from the Pacemaker point of view.

Here there is a crm\_mon output where both nodes are online:
\begin{verbatim}
============
Last updated: Sun Sep  8 13:06:11 2013
Last change: Sun Sep  8 13:04:19 2013 via crmd on primary
Stack: openais
Current DC: primary - partition with quorum
Version: 1.1.6-9971ebba4494012a93c03b40a2c58ec0eb60f50c
2 Nodes configured, 2 expected votes
0 Resources configured.
============

Online: [ secondary primary ]
\end{verbatim}

\section {bTactic Zimbra OCF installation}
\textbf{In both nodes} we need to obtain the Zimbra OCF script. Zimbra OCF script is found inside BtacticOCF tar.gz file (\cite{BtacticOCF}) which can be downloaded from BtacticOCF  tar.gz file (\cite{BtacticOrg}).

From the tar.gz we will use the zimbra script which we will copy into:
\begin{verbatim}
/usr/lib/ocf/resource.d/btactic
\end{verbatim}

We will use a temporary directory in order to use it:
\begin{verbatim}
mkdir /tmp/temp
cd /tmp/temp
\end{verbatim}
We download and extract it:
\begin{verbatim}
wget "http://www.btactic.org/btactic_ocf_0.0.2.tar.gz"
tar xzf btactic_ocf_0.0.2.tar.gz
\end{verbatim}
Make the btactic resource directory and copy zimbra file in there:
\begin{verbatim}
mkdir --parents /usr/lib/ocf/resource.d/btactic
cp zimbra /usr/lib/ocf/resource.d/btactic
\end{verbatim}
We finally make sure to give the script executable permissions:
\begin{verbatim}
chmod +x /usr/lib/ocf/resource.d/btactic/*
\end{verbatim}

% Pacemaker Final Setup

\section {\label{sec:pacemaker-final-setup}Pacemaker final setup}
As explained in \textit{section \ref{sec:about-pacemaker} About Pacemaker} Pacemaker is High Availability resource manager, here we will explain how our setup pretends to manage our two servers resources. These instructions should only be performed on \textbf{primary} node.

We need to introduce resource stickiness concept. Resource stickiness controls how much a service prefers to stay running where it is. You may like to think of it as the \textit{cost} of any downtime. By default, Pacemaker assumes there is zero cost associated with moving resources and will do so to achieve \textit{optimal} resource placement (\cite{ClustersFromScratch}).

These are the main settings we define in our configuration:
\begin{itemize}
  \item Setup deletes prior configuration.
  \item DRBD, Filesystem mount and Zimbra Server are setup to work in the same server as they work as in a team.
  \item System stickiness is changed so that Zimbra Server resource is not moved from where it is running to avoid Zimbra unnecessary downtimes.
  \item System are forced to be started in the right order. The right order is: DRBD, Filesystem mount, and Zimbra Server.
  \item We disable stonith (definition on subsection {\ref{subsec:fencing} Fencing}) in order to simplify the setup.
  \item Primary server will be the preferred server where resources need to be run.
\end{itemize}

We make sure that \textit{/tmp/zimbrapacemaker.config} file contents are:
\begin{verbatim}
configure
erase
node primary
node secondary
# Activate failover
property no-quorum-policy=ignore
# Disable stonith
property stonith-enabled=false
# Resource stickiness
rsc_defaults resource-stickiness=100
# Public ip fail over check
primitive ClusterIP ocf:heartbeat:IPaddr2 \
params nic=eth0 ip=192.168.77.203 \
cidr_netmask=24 \
broadcast=192.168.77.255 \
op monitor interval=30s
# Configure zimbra resource
primitive ZimbraServer ocf:btactic:zimbra op \
monitor interval=2min timeout="40s" \
op start interval="0" timeout="360s" \
op stop interval="0" timeout="360s"
# Prefered location: primary node
location prefer-primary-node \
ZimbraServer 50: primary
# Define DRBD ZimbraData
primitive ZimbraData ocf:linbit:drbd params \
drbd_resource=zimbradata op monitor \
role=Master interval=60s op monitor \
role=Slave interval=50s \
op start role=Master interval="0" timeout="240" \
op start role=Slave interval="0" timeout="240" \
op stop role=Master interval="0" timeout="100" \
op stop role=Slave interval="0" timeout="100"
# Define DRBD ZimbraData Clone
ms ZimbraDataClone ZimbraData meta \
master-max=1 master-node-max=1 \
clone-max=2 clone-node-max=1 notify=true
# Define ZimbraFS so that zimbra can use it
primitive ZimbraFS ocf:heartbeat:Filesystem \
params device="/dev/drbd/by-res/zimbradata" \
directory="/opt" fstype="ext4" \
op start interval="0" timeout="60s" \
op stop interval="0" timeout="60s"
group MyZimbra ZimbraFS ZimbraServer


# Everything in the same host
colocation everything-together \
inf: MyZimbra \
ZimbraDataClone:Master ClusterIP

# Everything ordered
order everything-ordered \
inf: \
ClusterIP \
ZimbraDataClone:promote MyZimbra
commit
\end{verbatim}

In order to apply the configuration we will run:
\begin{verbatim}
crm < /tmp/zimbrapacemaker.config
\end{verbatim}

Pacemaker configuration is not trivial. The main document from which the configuration file was adapted and written was Clusters from Scratch (\cite{ClustersFromScratch}).